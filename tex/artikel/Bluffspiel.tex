\section{Das große Bluffspiel}
\begin{pullquote}{shape=image, image=private/res/bluff_chips_karten.jpg, imageopts={width=8.4cm}}
% LaTeX-Warnungen vermeiden
\hbadness=3000
An dieser Stelle möchte ich einmal ein paar Worte über eine weit verbreitete Unart an der Hochschule verlieren: "Den großen Bluff".
Was um alles in der Welt ist das?
Nun, du wirst es in den Vorlesungen und Übungen erleben: Da sitzen um dich herum lauter hochintelligente und urgescheite Leute, die aufmerksam den Profs lauschen und all das verstehen, woran du selber fast verzweifelst.
Ähnlich wie beim Poker: Es gewinnt, wer das coolste und undurchschaubarste Gesicht macht, ohne etwas auf der Hand zu haben (bzw.\ in diesem Fall: ohne ein Wort verstanden zu haben).\pullquotenl

Manche Studierende beherrschen das "Bluffspiel" bis zur Vollendung.
Das kann gerade am Anfang ganz schön fertigmachen, wenn man den Eindruck hat, dass alle anderen den Stoff schon längst kapiert haben, und dass doch eigentlich alles ganz trivial sein muss.
Du willst dir keine Blöße geben und fängst an, bei dem Spiel mitzuspielen und nach kurzer Zeit beherrschst du es auch, doch hilft dies leider niemandem.
Nach einer gewissen Zeit studieren dann alle alleine nebeneinander her -- niemand traut sich, andere um Hilfe zu bitten: "Es könnte ja jemand merken, dass ich nicht viel weiß."%\pullquotenl

Wenn es einmal so weit gekommen ist, ist es unwahrscheinlich schwierig, mit dem "Bluffspiel" wieder aufzuhören.
Dabei könnte es viel einfacher sein, wenn alle ein bisschen auf die anderen zugehen würden.
Die selbstsichere Maske deiner Kommilitonen und Kommilitoninnen ist nämlich meistens nur aufgesetzt, um die eigene Unsicherheit zu vertuschen.
Wenn man die Distanz einmal überwunden hat, wird man feststellen, dass es den anderen genauso geht, wie einem selbst.
Meist sind sie dann ganz froh, dass jemand auf sie zugekommen ist, was sie selbst vielleicht nicht geschafft haben und sie sich nun einmal so zeigen, wie sie wirklich sind.
Das Studium ist wirklich viel leichter zu bewältigen, wenn man Freundschaften hat, mit denen man lachen, weinen, lernen oder über Probleme sprechen kann.
Wenn alle etwas mehr zusammenhalten, macht es auch wesentlich mehr Spaß.
Deshalb bitte ich dich: Versuche nicht, dass Studium als Einzelkämpfer*in anzugehen und den anderen vorzuspielen, dass du keine Hilfe brauchst.
Du wirst auch ohne all das akzeptiert werden.\pullquotenl
\pullquotenl

\fibelsigpq{Andreas G.}
\pullquotenl
\end{pullquote}
