% XXX Jedes Jahr O-Wochen-Plan aktualisieren!
% Zeilenumbruch \\ als \fibnlx zwischenspeichern (in Tabellen wird \\ für eine
% neue Tabellenzeile umdefiniert)
\let\fibnlx=\\
% Befehl \fibnl ist ein Zeilenumbruch mit etwas Freiraum darunter
\DeclareDocumentCommand{\fibnl}{}{\fibnlx[\baselineskip]}
\DeclareDocumentCommand{\fibabstand}{}{\vspace{.5\baselineskip} }

% XXX Relevanz dieser Definition prüfen
% Nur für den Donnerstags-Einschub.
\newlength{\fibprogrammeinschub}
\setlength{\fibprogrammeinschub}{0.005\textheight}

% Länge \fibprogrammcw ist die Breite einer Spalte in der Programmtabelle
% (außer Spalte mit den Zeiten)
\newlength{\fibprogrammcw}
\setlength{\fibprogrammcw}{0.214\textheight - 0.25\fibprogrammeinschub}


\begin{landscape}
\vspace*{\fill}
\section{Programm der Orientierungswoche Physik (O-Woche/Ersti-Woche)}
\newcommand{\GEINS}{G1 \tikz\draw[black,fill=black, line width=0.5mm] (0,0) circle (.7ex);\,\,}
\newcommand{\GZWEI}{G2 \tikz\draw[black, line width=0.5mm] (0,0) circle (.7ex);\,\,}
\renewcommand{\arraystretch}{1.8}
\footnotesize
\begin{tabular}{
	|
	>{\bfseries\hfill} % füge „\bfseries\hfill“ zu Beginn jeder Zelle dieser Spalte ein (also fett und rechtsbündig)
	p{0.08\textheight} % Textspalte (mehrzeilig) der Breite 0.08\textheight
	|
	*{4} % füge die folgende Definition 3x ein
	{
		p{\fibprogrammcw} % Textspalte (mehrzeilig) der Breite \fibprogrammcw
		|
	}
}
\hline
%%
%% TABLE HEADER
%%

Uhrzeit &
	\textbf{Montag, 26.10., \GEINS u. \GZWEI} &
	\textbf{Dienstag, 27.10., \GEINS} &
	\textbf{Mittwoch, 28.10., \GEINS u. \GZWEI} &
	\textbf{Donnerstag, 29.10., \GZWEI}
%% Horizontal Lines
\\ \cline{1-5}
%%
%%  09:30
%%
9:30\fibabstand\fibabstand\fibabstand &
% Mo
	& 
% Di
	&
% Mi
	\multirow{3}{\fibprogrammcw}[-2mm]{
            \textbf{Ersti-Begrüßung des Rektorats}\fibnlx
            \textit{\GEINS online, \GZWEI präsenz u. online}\fibnl
		\textbf{Einführungsveranstaltung}\fibnl
		\textbf{Tutorien und Institutsführung}\fibnl
		\textbf{Mittagspause}
		} & 
% Do
%%
%%  10:00
%%
\\ \cline{1-3}\cline{5-5}
10:00\fibabstand\fibabstand\fibabstand\fibabstand\fibabstand\fibabstand\fibabstand &
% Mo
	\multirow{2}{\fibprogrammcw}[-2mm]{%
		\textbf{Einführungsveranstaltung}\fibnlx
		\textit{\GEINS}\fibnl
		\textbf{Tutorien und Institutsführung}\fibnl
		\textbf{Mittagspause}
	} & 
% Di
		\textbf{Infoveranstaltung~II}\fibnlx
		\textit{\GEINS}\fibnl
		(Gremien, IVV~NWZ+ZIV, BAföG)
    &
% Mi
	&
% Do
    \textbf{Infoveranstaltung~II}\fibnlx
    \textit{\GZWEI}\fibnl
    (Gremien, IVV~NWZ+ZIV, BAföG)
%% Horizontal Lines
\\ \cline{1-1}\cline{3-3}\cline{5-5}
%%
%%  12:00
12:00 \fibabstand &
%% Mo
    &
% Di
    \textbf{Mittagspause} &
% Mi
    & 
% Do
    \textbf{Mittagspause}
%% Horizontal Lines
\\ \cline{1-5}
%%
%%  13:00
13:00 \fibabstand& 
%% Mo
\multirow{2}{\fibprogrammcw}[-2mm]{
        \textbf{Infoveranstaltung~I}\fibnlx
        \textit{\GEINS}\fibnl
        (Bachelor Physik, Zwei-Fach-Bachelor) 
	}&
% Di
    \textbf{Buch-Club}\fibnlx
    \textit{\GEINS}\fibnl
	(Fachliteratur) &
% Mi
\multirow{2}{\fibprogrammcw}[-2mm]{
    \textbf{Infoveranstaltung~I}\fibnlx
    \textit{\GZWEI}\fibnl
    (Bachelor Physik/Geophysik, Zwei-Fach-Bachelor) 
    } &
% Do
    \textbf{Buch-Club}\fibnlx
    \textit{\GZWEI}\fibnl
	(Fachliteratur)
%% Horizontal Lines
\\ \cline{1-1}\cline{3-3}\cline{5-5}
%%
%%  14:00
%%
14:00 \fibabstand &
% Mo
	&
% Di
	\multirow{3}{\fibprogrammcw}[4mm]{\textbf{Stadtspiel}\fibnlx
    \textit{\GEINS, nur Präsenz}
	}&
% Mi
	& 
% Do
	\multirow{3}{\fibprogrammcw}[4mm]{\textbf{Stadtspiel}\fibnlx
    \textit{\GZWEI, nur Präsenz}
	}
%% Horizontal Lines
\\ \cline{1-2}\cline{4-4}
%%
%%  15:00
%%
15:00 \fibabstand &
% 
% Mo & Di & Mi & Do
 & & &
\\ \cline{1-2}\cline{4-4}
%%
%%  ab 19:00
%%
ab 19:00 \fibabstand &
% Mo
    \textbf{Online-Party}\fibnlx
    \textit{\GEINS, \GZWEI}
    &
% Di
    &
% Mi
    \textbf{Online-Kneipentour}\fibnlx
    \textit{\GEINS, \GZWEI}&
% Do
\\ \hline
\end{tabular}

\smallskip

% XXX Jedes Jahr Astroseminar-Termin aktualisieren!
\textbf{\GEINS: Gruppe "Montag/Dienstag", \GZWEI: Gruppe "Mittwoch/Donnerstag". Solange nicht anders beschriftet, finden Veranstaltungen sowohl in den Hörsälen als auch online statt. Mehr Infos zu den Orten gibt es in der Online-Reservierungsbestätigung.}
\vfill
\end{landscape}
