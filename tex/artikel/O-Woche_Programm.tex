% XXX Jedes Jahr O-Wochen-Plan aktualisieren!
% Zeilenumbruch \\ als \fibnlx zwischenspeichern (in Tabellen wird \\ für eine
% neue Tabellenzeile umdefiniert)
\let\fibnlx=\\
% Befehl \fibnl ist ein Zeilenumbruch mit etwas Freiraum darunter
\DeclareDocumentCommand{\fibnl}{}{\fibnlx[\baselineskip]}
\DeclareDocumentCommand{\fibabstand}{}{\vspace{.5\baselineskip}}

% XXX Relevanz dieser Definition prüfen
% Nur für den Dienstags-Einschub.
\newlength{\fibprogrammeinschub}
\setlength{\fibprogrammeinschub}{0.005\textheight}

% Länge \fibprogrammcw ist die Breite einer Spalte in der Programmtabelle
% (außer Spalte mit den Zeiten)
\newlength{\fibprogrammcw}
\setlength{\fibprogrammcw}{0.214\textheight - 0.25\fibprogrammeinschub}


\begin{landscape}
\section{Programm der Physik-Orientierungseinheit (O-Woche/Ersti-Woche)}
\renewcommand{\arraystretch}{1.8}
\footnotesize
\begin{tabular}{
	|
	>{\bfseries\hfill} % füge „\bfseries\hfill“ zu Beginn jeder Zelle dieser Spalte ein (also fett und rechtsbündig)
	p{0.08\textheight} % Textspalte (mehrzeilig) der Breite 0.08\textheight
	|
	*{5} % füge die folgende Definition 2x ein
	{
		p{0.8\fibprogrammcw} % Textspalte (mehrzeilig) der Breite \fibprogrammcw
		| % Breite der Spalten 
	}
	% Dienstags-Einschub.
% 	p{\fibprogrammeinschub} % Textspalte (mehrzeilig) der Breite \fibprogrammeinschub
% 	|
% 	*{2} % füge die folgende Definition 2x ein
% 	{
% 		p{\fibprogrammcw} % Textspalte (mehrzeilig) der Breite \fibprogrammcw
% 		|
% 	}
}
\hline
%%
%% TABLE HEADER
%%
Uhrzeit &
	\textbf{Montag, 04.10.} &
	\textbf{Dienstag, 05.10.} &
% 	\multirow{10}{*}{
% 		\hspace*{-2mm}\rotatebox{-90}{\hspace*{.07cm} \textbf{Mittwoch} ab 13 Uhr\textbf{: Gemütliches Beisammensein in kleinerem Rahmen}}
% 	}&
	\textbf{Mittwoch, 06.10.} &
	\textbf{Donnerstag, 07.10.} &
	\textbf{Freitag, 08.10.}
%% Horizontal Lines
% \\ \cline{1-3}\cline{5-6}
\\ \hline 
%%
%%  10:00
%%
10:00\fibabstand\fibabstand\fibabstand &
% Mo
	\multirow{3}{0.8\fibprogrammcw}[-2mm]{%
		\textbf{Einführungsveranstaltung}\fibnlx
		\hspace*{\fill}
		\textit{Hörsaal~AP}\fibnl
		\textbf{Tutorien und Institutsführung}
	} & 
% Di 
	\multirow{2}[2]{0.8\fibprogrammcw}[-3mm]{%
		\textbf{Infoveranstaltung~II}\fibnlx
		(Gremien, IVV~NWZ+ZIV, BAföG)\fibnl
		\hspace*{\fill}
		\textit{Hörsaal~AP}} & 
% Mi		
		\multirow{2}{0.8\fibprogrammcw}[-3mm]{%
	    \textbf{Ausweichtermin Infoveranstaltung~I}\fibnlx
		(nur für Zwei-Fach-Bachelor)\fibnl
		\hspace*{\fill}
		\textit{Hörsaal~AP}}
	& 
% Do
	\multirow{2}[2]{0.8\fibprogrammcw}[-3mm]{%
	\textbf{Ersti-Begrüßung des Rektorats}\fibnl
	    (mit anschließender Taschenausgabe)\fibnl\fibnlx
		\hspace*{\fill}
		\textit{Preußenstadion} 
	}
	&
% Fr
    
%% Horizontal Lines
\\ \cline{1-1} \cline{6-6}

%%
%%  11:00
%% Mo & Di & Mi & Do
11:00 \fibabstand & & & & &
% Fr
    \textbf{Vortrag der Polizei}\fibnl
		\hspace*{\fill}\textit{Hörsaal~1}

%% Horizontal Lines
\\ \cline{1-4}\cline{6-6}

%%
%%  12:00
%% Mo
12:00 \fibabstand & 
		\textbf{Mittagessen}\fibnl
    &
% Di 
	\textbf{Buch-Club}\fibnlx
	(Fachliteratur)\fibnlx
	\hspace*{\fill}
	    \textit{Hörsaal~AP} 
    & 
% Mi	
	\textbf{Laborführungen}\fibnl
		\hspace*{\fill}
		\textit{Treffen vor der Fachschaft} 
	&
% Do 
    &
% Fr
    \multirow{2}[2]{0.8\fibprogrammcw}{%
		\textbf{Mittagspause}\fibnl
		\hspace*{\fill}
	}
%% Horizontal Lines
\\ \cline{1-4}\cline{6-6}

%%
%%  13:00
%% Mo
13:00 \fibabstand\fibabstand\fibabstand & 
    \multirow{2}[2]{0.8\fibprogrammcw}[-2mm]{%
        \textbf{Infoveranstaltung~I}\fibnlx
		(Bachelor Physik/Geophysik, Zwei-Fach-Bachelor)\fibnl
		\hspace*{\fill}
		\textit{Hörsaal~AP}
	}
	&
% Di 
	\multirow{2}[2]{0.8\fibprogrammcw}{%
		\textbf{Mittagspause}\fibnl
		\hspace*{\fill}
	}
	& 
% Mi
	\multirow{2}[2]{0.8\fibprogrammcw}{%
		\textbf{Mittagspause}\fibnl
		\hspace*{\fill}
	}
	&
% Do
	
    &
% Fr
	\multirow{2}[15]{0.8\fibprogrammcw}{\textbf{Infoveranstaltung~III}\fibnlx
		(jDPG \& andere studentische Gruppierungen)\fibnlx[0.5em]
		\textbf{Plenum \& Preisverleihung}\fibnlx
		\hspace*{\fill}
		\textit{Hörsaal~1}}
		
%% Horizontal Lines
\\ \cline{1-1}\cline{3-3}\cline{5-5}

%%
%%  14:00
%%
14:00 \fibabstand\fibabstand\fibabstand\fibabstand & 
% Mo
    & 
% Di & Mi
	\multirow{2}[2]{0.8\fibprogrammcw}{\textbf{Stadtspiel}\fibnl
		\hspace*{\fill}
		\textit{Treffen an der Freitreppe}} & & 
% Do
    \multirow{2}[2]{0.8\fibprogrammcw}{%
		\textbf{Mittagspause}\fibnl
		\hspace*{\fill}
	}
 
    &
% Fr
	
%% Horizontal Lines
\\ \cline{1-2}\cline{5-6}
%%
%%  15:00
%%
15:00 \fibabstand &
% Mo
    \multirow{4}[8]{0.8\fibprogrammcw}{\textbf{Campusspiel}\fibnl
		\hspace*{\fill}
		\textit{Treffen an der Freitreppe}\fibnlx\fibnlx\fibnlx
		anschließend Grillen vor der Fachschaft\fibnlx
		(bei passender Wetterlage)}
	& 
% Di - Fr	
	& & 
	
	\multirow{2}[2]{0.8\fibprogrammcw}{%
		\textbf{Konstruktionswettbewerb}\fibnl
		\hspace*{\fill}
		\textit{Treffen vor der Fachschaft}
	}
	&
	\textbf{Restegrillen}
\\ \cline{1-1}\cline{4-4}
%%
%%  16:00
%%
\diagbox[width=4.76em]{Mi,\\16:30}{16:00} & %\fibabstand &
% Mo 
	 & 
% Di		
		& 
		
% Mi
	\multirow{2}[6]{0.8\fibprogrammcw}{%
		\textbf{"Kaffeetrinken" mit den Professoren}\fibnl
		\hspace*{\fill}
		\textit{Foyer IG1}
	}&
% Do 
% 	\multirow{2}[2]{0.8\fibprogrammcw}{%
% 		\textbf{Konstruktionswettbewerb}\fibnl
% 		\hspace*{\fill}
% 		\textit{Treffen im Foyer IG1}
% 	}
    &
% Fr 
%% Horizontal Lines
\\ \cline{1-1}
%%
%%  17:00
%%
17:00 \fibabstand & & & & &
%% Horizontal Lines
\\ \cline{1-1}\cline{4-5}
%%
%%  18:00
%% Mo
18:00 \fibabstand &	&
% Di 
% 	\multirow{2}{0.8\fibprogrammcw}{%
% 		\textbf{Grillen}\fibnlx
% 		(bei passender Wetterlage)
% 	} 
    & 

% Mi	
	&
% Do
	\multirow{3}{0.8\fibprogrammcw}{%
		\textbf{Spieleabend}\fibnlx
		Brettspiele, Kartenspiele, Pen~\&~Paper und mehr!\fibnl
		\hspace*{\fill}
		\textit{Seminarraum KP~104}
	} &
% Fr

\\ \cline{1-1}\cline{3-3}
%%
%%  19:00
%%
19:00 \fibabstand &	&
% Di 
\textbf{Kneipenabend}\fibnlx
		\hspace*{\fill}
		\textit{im "La~Palma"}
& & & 
%% Horizontal Lines
\\ \hline 

\end{tabular}

\smallskip

% XXX Jedes Jahr Astroseminar-Termin aktualisieren!
\textbf{Du findest Astrophysik spannend?
	Zum Thema "Von Geisterteilchen und kosmischen Singularitäten" findet dieses Jahr wieder das jährliche \textit{Astroseminar} am Fr.\ und Sa., 15.10. \& 16.10. statt.
	Eintritt frei -- Anmeldung und Hygieneregeln sowie den 3G-Nachweis beachten! \\
	Mehr Infos unter \url{https://www.uni-muenster.de/Astroseminar}
	}
\end{landscape}
