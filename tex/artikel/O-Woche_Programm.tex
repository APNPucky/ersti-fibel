% XXX Jedes Jahr O-Wochen-Plan aktualisieren!
% Zeilenumbruch \\ als \fibnlx zwischenspeichern (in Tabellen wird \\ für eine
% neue Tabellenzeile umdefiniert)
\let\fibnlx=\\
% Befehl \fibnl ist ein Zeilenumbruch mit etwas Freiraum darunter
\DeclareDocumentCommand{\fibnl}{}{\fibnlx[\baselineskip]}
\DeclareDocumentCommand{\fibabstand}{}{\vspace{.5\baselineskip}}

% XXX Relevanz dieser Definition prüfen
% Nur für den Dienstags-Einschub.
\newlength{\fibprogrammeinschub}
\setlength{\fibprogrammeinschub}{0.005\textheight}

% Länge \fibprogrammcw ist die Breite einer Spalte in der Programmtabelle
% (außer Spalte mit den Zeiten)
\newlength{\fibprogrammcw}
\setlength{\fibprogrammcw}{0.268\textheight - 0.25\fibprogrammeinschub}


\begin{landscape}
\section{Programm der Physik-Orientierungseinheit (O-Woche/Ersti-Woche)}
\renewcommand{\arraystretch}{1.8}
\footnotesize
\begin{tabular}{
	|
	>{\bfseries\hfill} % füge „\bfseries\hfill“ zu Beginn jeder Zelle dieser Spalte ein (also fett und rechtsbündig)
	p{0.08\textheight} % Textspalte (mehrzeilig) der Breite 0.08\textheight
	|
	*{5} % füge die folgende Definition 2x ein
	{
		p{0.8\fibprogrammcw} % Textspalte (mehrzeilig) der Breite \fibprogrammcw
		| % Breite der Spalten 
	}
	% Dienstags-Einschub.
% 	p{\fibprogrammeinschub} % Textspalte (mehrzeilig) der Breite \fibprogrammeinschub
% 	|
% 	*{2} % füge die folgende Definition 2x ein
% 	{
% 		p{\fibprogrammcw} % Textspalte (mehrzeilig) der Breite \fibprogrammcw
% 		|
% 	}
}
\hline
%%
%% TABLE HEADER
%%
Uhrzeit &
	\textbf{Montag, 02.10.} &
	\textbf{Mittwoch, 04.10.} &
% 	\multirow{10}{*}{
% 		\hspace*{-2mm}\rotatebox{-90}{\hspace*{.07cm} \textbf{Mittwoch} ab 13 Uhr\textbf{: Gemütliches Beisammensein in kleinerem Rahmen}}
% 	}&
	\textbf{Donnerstag, 05.10.} &
	\textbf{Freitag, 06.10.} 
%% Horizontal Lines
\\ \hline 
%%
%%  10:00
%%
10:00 \fibabstand\fibabstand\fibabstand &
% Mo
	\multirow{2}{0.8\fibprogrammcw}[-2.3mm]{%
		\textbf{Einführungsveranstaltung}\fibnl\fibnlx
		\hspace*{\fill}
		\textit{Hörsaal~2}
	}
	& 
% Mi 
    \multirow{2}{0.8\fibprogrammcw}[-2.5mm]{%
		\textbf{Infoveranstaltung~II}\fibnlx
		(Gremien, IVV + WWU~IT, BAföG)\fibnl
		\hspace*{\fill}
		\textit{Hörsaal~2}
	} 
% 	\multirow{2}{0.8\fibprogrammcw}[-1mm]{%
% 	    \textbf{Ausweichtermin Infoveranstaltung~I}\fibnlx
% % 		(nur für Zwei-Fach-Bachelor)\fibnl\fibnlx
% 		\hspace*{\fill}
% 		\textit{Hörsaal~1}
% 	} 
	&
% Do
	% \textbf{Laborführungen}\fibnl
	% 	\hspace*{\fill}
	% 	\textit{Treffen um 9:45~Uhr an der Freitreppe}
	& 
% Fr
    \textbf{Laborführungen}\fibnl
		\hspace*{\fill}
		\textit{Treffen um 10:15~Uhr im Foyer der IG~I}
%% Horizontal Lines
\\ \hline

%%
%%  11:00
%%
\diagbox[width=4.76em]{Mi,\\11:30}{11:00} & %\fibabstand 
% 11:00 \fibabstand & 
% Mo
    \multirow{2}[2]{0.8\fibprogrammcw}[11mm]{%
        \textbf{Tutorien und Institutsführung}
	    \hspace*{\fill}
    }
    & 
% Mi 
    \textbf{Mittagessen}
	\hspace*{\fill}
    & 
% Do
    \multirow{2}{0.8\fibprogrammcw}[5.5mm]{%
	    \textbf{Ersti-Begrüßung des Rektorats}\fibnlx
	    (mit anschließender Taschenausgabe)\fibnl\fibnlx
		\hspace*{\fill}
		\textit{Preußenstadion} 
	}

 %    \multirow{2}{0.8\fibprogrammcw}[-2.5mm]{%
	% 	\textbf{Infoveranstaltung~II}\fibnlx
	% 	(Gremien, IVV + WWU~IT, BAföG)\fibnl
	% 	\textbf{+~Buchclub}\fibnlx
	% 	\hspace*{\fill}
	% 	\textit{Hörsaal~1}
	% } 
	& 
% Fr 
    % \textbf{Vortrag der Polizei}\fibnl
	% \hspace*{\fill}
    % \textit{Hörsaal~1}
%% Horizontal Lines
\\ \hline

%%
%%  12:00
%% 
\diagbox[width=4.76em]{Mi,\\12:30}{12:00} & %\fibabstand 
% 12:00 \fibabstand & 
% Mo
	\multirow{2}[2]{0.8\fibprogrammcw}[11mm]{%
        \textbf{Mittagessen}
	    \hspace*{\fill}
    }
    &
% Mi
    \multirow{2}[2]{0.8\fibprogrammcw}[-2.3mm]{%
		\textbf{Stadtspiel}\fibnl
		\hspace*{\fill}
		\textit{Treffen hinter der IG~I}
	}
    &
% Do
    \multirow{2}{0.8\fibprogrammcw}[4.8mm]{%
	    \textbf{Ersti-Messe}\fibnl
	    \hspace*{\fill}
		\textit{Preußenstadion}\fibnl
        alternativ: \textbf{Ausweichtermin Infoveranstaltung~I}\fibnl
        \hspace*{\fill}
        \textit{Hörsaal~1}
	}
	&
% Fr 
    \multirow{2}[15]{0.8\fibprogrammcw}[5.5mm]{%
	    \textbf{Infoveranstaltung~III}\fibnlx
		(jDPG \& andere studentische Gruppierungen)\fibnl
		\textbf{Plenum \& Preisverleihung}\fibnl
		\hspace*{\fill}
		\textit{Hörsaal~1}
	}
%% Horizontal Lines
\\ \cline{1-2}

%%
%%  13:00
%%
13:00 \fibabstand\fibabstand\fibabstand & 
% \diagbox[width=4.76em]{Do,\\13:30}{13:00} & %\fibabstand 
%% Bei Diag-Box: 
%% Für Info 1 und Info 3: [4.5mm] (Info 1) und [6mm] (Info 3), sonst: [-2.5mm] (Info 1) und keine (Info 3)!
%% Für Mi: [11mm] (Mittagessen) 
% Mo    
    \multirow{2}[2]{0.8\fibprogrammcw}[-2.5mm]{%
        \textbf{Infoveranstaltung~I}\fibnlx
		(Bachelor Physik/Geophysik, Zwei-Fach-Bachelor)\fibnl
		\hspace*{\fill}
		\textit{Hörsaal~1, 2 und Seminarraum der Geophysik}
	}
	&
% Mi 
    % \multirow{2}[2]{0.8\fibprogrammcw}[11mm]{%
    %     \textbf{Mittagessen}
	   %  \hspace*{\fill}
    % }
	& 
% Do
	% \textbf{Mittagessen}
	% \hspace*{\fill}
	&
% Fr
	% \multirow{2}[15]{0.8\fibprogrammcw}[5.5mm]{%
	%     \textbf{Infoveranstaltung~III}\fibnlx
	% 	(jDPG \& andere studentische Gruppierungen)\fibnl
	% 	\textbf{Plenum \& Preisverleihung}\fibnl
	% 	\hspace*{\fill}
	% 	\textit{Hörsaal~1}
	% 	}
%% Horizontal Lines
\\ \cline{1-1}\cline{4-5}

%%
%%  14:00
%%
% 14:00 \fibabstand\fibabstand\fibabstand\fibabstand & 
14:00 \fibabstand &
%% Für Mi: [6.5mm] (Stadtspiel), sonst keine! 
% Mo
    & 
% Mi 
	% \multirow{2}[2]{0.8\fibprogrammcw}[4.8mm]{%
	% 	\textbf{Stadtspiel}\fibnl
	% 	\hspace*{\fill}
	% 	\textit{Treffen hinter der IG~I}\fibnlx\fibnlx\fibnlx
	% 	anschließend Grillen auf der Wiese hinter der IG~I\fibnlx
	% 	(bei passender Wetterlage)
	% }
	& 
% Do 
    \multirow{2}[2]{0.8\fibprogrammcw}{%
		\textbf{Konstruktionswettbewerb}\fibnl
		\hspace*{\fill}
		\textit{Treffen an der Freitreppe}
	}
	& 
% Fr
    \textbf{Restegrillen}\fibnlx[0.5em]
	\textbf{+~Professoren-Vorstellung}
	\hspace*{\fill}
%% Horizontal Lines
\\ \cline{1-2}

%%
%%  15:00
%%
15:00 \fibabstand &
% Mo
    \multirow{2}[2]{0.8\fibprogrammcw}[-2.3mm]{%
		\textbf{Campusspiel}\fibnl
		\hspace*{\fill}
		\textit{Treffen hinter der IG~I}\fibnlx\fibnlx\fibnlx
        anschließend Grillen auf der Wiese hinter der IG~I\fibnlx
		(bei passender Wetterlage)
	} 
	& 
% Mi
%     \multirow{4}[8]{0.8\fibprogrammcw}[-2.3mm]{%
%         \textbf{Campusspiel}\fibnl
% 		\hspace*{\fill}
% 		\textit{Treffen hinter der IG~I}\fibnlx\fibnlx\fibnlx
% 		anschließend Grillen auf der Wiese hinter der IG~I\fibnlx
% 		(bei passender Wetterlage)\fibnl\fibnlx\fibnlx
% 		abends abbauen
% 		}
	& 
% Do 
	& 
% Fr 
	% \textbf{Restegrillen}\fibnlx[0.5em]
	% \textbf{+~Professoren-Vorstellung}
	% \hspace*{\fill}
%% Horizontal Lines 
\\ \cline{1-1}

%%
%%  16:00
%%
16:00 \fibabstand & 
% Mo 
	& 
% Mi		
	& 
		
% Do
% 	\multirow{3}[6]{0.8\fibprogrammcw}{%
% 		\textbf{"Kaffeetrinken" mit den Profs}\fibnl
% 		\hspace*{\fill}
% 		\textit{Treffpunkt je nach Wetterlage, siehe Website}\fibnl\fibnlx
% 		anschließend entspannter Abend mit Flunky-Ball und Essen bestellen
% 	} 
	&
% Fr 
%% Horizontal Lines
\\ \cline{1-1}

%%
%%  17:00
%%
17:00 \fibabstand & 
% Mo
& 
% Mi 
& 
% Do 
& 
% Fr 
%% Horizontal Lines
\\ \cline{1-1}\cline{4-4}

%%
%%  18:00
%%
18:00 \fibabstand &	
% Mo
  %   \textbf{Kneipenabend}\fibnlx[0.5em]
		% \hspace*{\fill}
		% \textit{im ,,Das Piano''}
    &
% Mi 
    & 
% Do	
	\multirow{3}{0.8\fibprogrammcw}[-0.6mm]{%
		\textbf{Spieleabend}\fibnlx
		Brettspiele, Kartenspiele, Pen~\&~Paper und mehr!\fibnlx[0.58em]
		\hspace*{\fill}
		\textit{div. Räume der KP}
	} 
	&
% Fr
%% Horizontal Lines 
\\ \cline{1-1}\cline{3-3}

%%
%%  19:00
%%
19:00 \fibabstand &	
% Mo
    abends abbauen
    \hspace*{\fill}
    &
% Mi 
    \textbf{Kneipenabend}\fibnlx[0.5em]
		\hspace*{\fill}
		\textit{in der Cavete}
    & 
% Do 
    & 
% Fr 
    abends abbauen
    \hspace*{\fill}
%% Horizontal Lines
\\ \hline 

\end{tabular}

\smallskip

% XXX Jedes Jahr Astroseminar-Termin aktualisieren!
% \textbf{Das aktuellste Programm und kurzfristige Raumankündigungen findet ihr unter \url{https://www.uni-muenster.de/Physik.FSPHYS/studieren/erstis/woche/}.} \\ 

\textbf{%
    Du findest Astrophysik spannend?
	Dann schau beim jährlichen \textit{Astroseminar} vorbei! Es findet dieses Jahr am Fr.\ und Sa., den 13.10. \& 14.10.2023 im Hörsaal~1 statt. Eintritt frei. Weitere Infos unter \url{https://www.uni-muenster.de/Astroseminar}. 
	}
	
\end{landscape}
