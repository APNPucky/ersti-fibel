\section[Studieren im Ausland – Erasmus]{Studieren im Ausland – Das Erasmus-Programm}
\begin{multicols}{2}
\textit{Im Zusammenhang mit dem Studium hast du bestimmt schon häufiger gehört, dass im Laufe dessen Viele ein Semester oder Jahr im Ausland studieren oder vielleicht sogar ein ganzes Master-Studium in einem anderen Land abschließen.
Das klingt alles sehr aufregend, allerdings bestimmt auch ein bisschen beängstigend, wie am Anfang ja eigentlich auch das ganze Studium an sich!
Deshalb möchte ich euch hier ein wenig darüber erzählen – als jemand, der ein Jahr in Schweden (leider!) schon hinter sich hat.}

Der bei weitem unkomplizierteste Weg, während eines "normalen" Studiums in Deutschland einige Zeit (in der Physik in Münster üblicherweise ein Jahr) im Ausland zu verbringen, ist das Erasmus-Programm ("\textbf{E}uropean \textbf{r}egion \textbf{a}ction \textbf{s}cheme for the \textbf{m}obility of \textbf{u}niversity \textbf{s}tudents").
Das Erasmus-Programm ermöglicht es dir, überall in Europa relativ unbürokratisch zu studieren.
Auch mein Auslandsaufenthalt wurde durch dieses Programm ermöglicht.
Der Vorteil ist (neben der finanziellen Förderung), dass im Voraus bereits genau geklärt wird, wie dir die im Ausland erbrachten Studienleistungen nach der Rückkehr in Deutschland angerechnet werden.
Es kann also nicht zu bösen Überraschungen kommen.

\begin{quote}
	\textit{"Wieso sollte ich ins Ausland gehen?"}
\end{quote}
Das ist eine gute Frage.
Die Antwort darauf kannst du dir letzten Endes nur selbst geben.
Allerdings sind die fast ausnahmslos begeisterten Schilderungen derer, die dieses Wagnis eingegangen sind, vielleicht ein überzeugendes Argument, es zu versuchen – selbst dann, wenn sich das alles zuerst viel zu umständlich, anstrengend und aufwändig anhört, und man ja eigentlich gar nicht der Typ für sowas ist, und welche Ausreden sich das Unterbewusstsein sonst noch so ausdenkt ;-)

Als Erstes denkt man vermutlich an die fremde Sprache.
Das ist aber seltsamerweise kein Problem.
Obwohl ich beispielsweise bei meiner Ankunft in Lund kaum Schwedisch sprechen konnte, war die Kommunikation auf Englisch kein Problem.
Auch den (englischsprachigen) Vorlesungen konnte ich mühelos folgen.
Gerade in der Physik kommunizieren wir ja ohnehin über die "Sprache" der Mathematik.
Zudem gab es dort kostenlose Sprachkurse an der Uni, sodass ich nicht unwissend wieder zurückkam :-)

Tatsächlich war es gar nicht mal so einfach, die in den Kursen frisch erworbenen Kenntnisse im Alltag einzusetzen – wird man einmal von einem Schweden als Nicht-Muttersprachler ausgemacht, wird direkt auf Englisch gewechselt, was dort fast alle hervorragend beherrschen.

\begin{center}
	\includegraphics[width=\columnwidth]{res/erasmus_logo.pdf}
\end{center}

Als Nächstes könnte man einwenden, dass die Vorlesungen vielleicht nicht zu dem passen, was man machen möchte und muss.
Aber dieser Punkt darf getrost vergessen werden, da der Fachbereich Physik nur Kooperationen mit den Universitäten eingeht, bei denen der Studienverlaufsplan kompatibel mit unserem ist (ohnehin gibt es da im Bachelor-Studiengang zwischen den verschiedenen Universitäten nur eher geringfügige Unterschiede).
Dies ist auch ein Grund dafür, statt einem Semester ein ganzes Jahr zu bleiben, sodass man z.\,B.\ Quantenmechanik nicht doppelt hört.
Die Empfehlung ist, im 3.~Studienjahr (5./6.~Semester im Bachelor) zu gehen; man kann jedoch auch, so wie ich, im 1./2.~Master-Semester reisen.
Tatsächlich bin ich sogar gerade wegen dieser Unterschiede nach Lund gegangen, weil es dort Astronomie- und Astrophysik-Vorlesungen gibt, während dieser Bereich in Münster leider ziemlich eingeschränkt ist.

Weitere Bedenken kommen meist, wenn es um die Finanzierung geht.
Schweden beispielsweise ist in der Tat kein günstiges Land; Lebensmittel sind im Schnitt \SI{50}{\percent} teurer als hier. Es kommt aber auch auf die Stadt an, in der man lebt, denn Stockholm oder Malmö sind tendenziell teurer als Lund. Außerdem gibt es auch in Schweden günstige Supermärkte wie Lidl oder Willys.
Wenngleich die Mietpreise in Lund vergleichbar bzw.\ leicht höher liegen als in Münster, so muss man mit ca.~800\,€ im Monat kalkulieren, um locker über die Runden zu kommen.
Falls man keine reichen Eltern hat, ist das trotzdem kein Grund, auf das Ausland zu verzichten:
Die Förderung über das Erasmus ist eine Finanzierungsquelle, die dabei weiterhilft.
Je nachdem, wie hoch die Kosten im Zielland sind, erhält man eine unterschiedlich hohe Unterstützung (z.\,B.\ in Schweden, welches zur teuersten Kategorie gehört, zur Zeit 450\,€ im Monat).
Außerdem besteht Anspruch auf Auslands-BAföG, wobei der Satz höher liegt als beim Inlands-BAföG.
Falls du also hier keins bekommst, ist es trotzdem möglich, dass der Aufenthalt gefördert werden kann.

Aber den Hauptgrund, wieso es sich lohnt, kann ich nicht in einem Artikel beschreiben.
Man muss dieses Gefühl einmal selbst erlebt haben.
Wenn man sich unbekümmert mit benachbarten Indern über die Bedeutung von Familie unterhalten, ein typisch australisches Frühstück verzehren, mit wildfremden Kroaten zusammen Bier trinken, mit (fast) nackten Schweden auf Tischen tanzen, mit betrunkenen Franzosen über das eigene Land herziehen und bis nachts um 5~Uhr zutiefst philosophische und religiöse Themen mit Ungarinnen ausdiskutieren kann – oder sich 5~Minuten mit jemandem auf Englisch unterhalten kann, nur um festzustellen, dass ihr doch beide aus Deutschland kommt – dann weiß man, dass man im Ausland studiert hat.
Dieses unbändige Gefühl der Freiheit gepaart mit jugendlichem Idealismus und die Gewissheit, dass einem die Türen der Welt offen stehen, erlebt man vermutlich wirklich nur einmal im Leben bei genau dieser Gelegenheit.

Und wer jetzt sagt:
"Hey, das ist nichts für mich – ich bin anders drauf und will auch nicht ein ganzes Jahr nur Party machen."
Dem möchte ich sagen: Kein Problem!
Ein Auslandsstudium ist eine Erfahrung, die individueller nicht sein könnte.
Wer möchte, kann auch alles entspannt angehen – ich habe beispielsweise viele nordische Städte besichtigt, die schwedische und finnische Natur bewundert und auf einer Schneeschuhwanderung im finnischen Lappland jenseits des Polarkreises die Nordlichter gesehen.
Und glaub mir: Selbst, wenn du dir vornimmst, daraus eine ruhige Sache zu machen – ein Abenteuer wird das Ganze auf jeden Fall!

Erasmus ist mehr als ein Eintrag im Lebenslauf.
Erasmus ist ein Stück Lebenserfahrung, das du nicht mehr missen möchtest.
Alles ist anders.
Überall fällt dir etwas auf, das ein kleines bisschen anders ist, als man es aus Deutschland kennt.
Du fängst an, darüber nachzudenken, welche Unterschiede es gibt.
Welche Dinge hierzulande besser sind, welche dort.
Du lernst nicht nur das Land besser kennen, sondern auch dein Heimatland.
Und entgegen aller Sorgen und Ängste, die ich hatte, kann ich wirklich jedem wärmstens empfehlen, über seinen eigenen Schatten zu springen und das Wagnis "Ausland" einzugehen.

Falls dein Interesse geweckt wurde, kannst du auf die Informationsveranstaltung im 2.~Semester warten oder dich schon vorher bei uns in der Fachschaft bzw.\ im Internet erkundigen:
\begin{center}
	% URL wurde mit \href gemacht, um den Zeilenumbruch manuell einfügen zu können
	\href{https://www.uni-muenster.de/Physik/international/}{\textbf{\texttt{https://www.uni-muenster.de/\\Physik/international/}}}
\end{center}

Prof.~Kappes (\email{erasmus.physik@uni-muenster.de}), der Erasmus-Koordinator am Fachbereich Physik, ist auch sehr nett und berät jederzeit gerne zu Auslandsaufenthalten.

\fibelsig{Simon}
\end{multicols}
