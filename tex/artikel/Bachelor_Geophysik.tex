\section{Bachelor-Studiengang Geophysik}
\begin{multicols}{2}
Jedem, der schon einmal auf WG-Partys, Einführungskursen, Kennenlern-Spielen oder Ähnlichem herumstand, wird früher oder später die Frage gestellt "Und was studierst du?".
Gibt man als Antwort schlicht und einfach Geophysik, erntet man meist nur fragende Blicke und ein "Aha, interessant\dots\ Und was ist das?".
Hier kommen auch langjährige Geophysik-Studenten oftmals ins Grübeln und eine Antwort der Art "Wir gucken uns Erdbeben an und das Erdmagnetfeld und die Ozeane und so." ist für beide nicht sehr zufriedenstellend.
Was macht man also in der heutigen Zeit, wenn man ein Wort hört, das man nicht kennt? Man schaut natürlich bei Wikipedia nach.
Frei übersetzt von der englischen Seite findet man dort:
\vspace*{-0.5cm}
\begin{quote}
	\textit{Geophysik, eine Hauptdisziplin der Geowissenschaften und eine Teildisziplin der Physik, ist die Erforschung der gesamten Erde durch quantitative Untersuchung der physikalischen Eigenschaften.}
\end{quote}

Auch wenn dies nur eine der vielen gängigen Definitionen ist (und man sich hierüber auch streiten kann), kommen wir der Sache schon etwas näher.
Die Geophysik gehört also vom Forschungsobjekt her gesehen zu den Geo- oder Erdwissenschaften, von der Wissenschaftsmethodik her jedoch zur Physik.
Dies ist auch der Grund dafür, dass Geophysik in Deutschland teilweise in physikalischen (wie in Münster) als auch in geowissenschaftlichen Fachbereichen angesiedelt ist.

Es geht dabei um Fragestellungen, wie Erdbeben entstehen und wie sich seismische Wellen durch die Erde ausbreiten, was man dadurch über den inneren Aufbau der Erde lernen kann, warum die Erde in Japan häufiger und stärker bebt als in Deutschland, wie die Erde in ihrem Inneren aufgebaut ist, warum, woher und wohin sich die Erdplatten bewegen, wie der Wasserkreislauf der Meere entsteht, wie sich das Klima entwickelt, ob die Kompassnadel in Münster auch wirklich exakt nach Norden zeigt, ob man in Greven in 50~Jahren seinen Badeurlaub buchen kann, warum sich das Magnetfeld der Sonne alle 11~Jahre umpolt, wie schnell sich die Gletscher in der Antarktis bewegen und noch vieles mehr.

%\iffalse
\begin{tikzpicture}[remember picture, overlay]
	\node[inner sep=0, yshift=2.6cm] at (current page.center)
	{\includegraphics[width=5cm]{res/geophysik_erde.pdf}};
\end{tikzpicture}
\parshape=10
0cm \columnwidth
0cm \columnwidth
0cm \columnwidth
0cm \columnwidth
0cm \columnwidth
0cm 7.5cm
0cm 7.1cm
0cm 6.7cm
0cm 6.6cm
0cm 6.6cm
%\fi
Zu guter Letzt existieren die praktischen Anwendungen der Geophysik, die sich eigentlich immer darum drehen, wie man Ressourcen in der Erde aufspüren kann, ohne hineinzubohren (denn das ist unvorstellbar teuer).
Dabei kann es sich um Öl, Gas, Grundwasser, Salz, Erze, oder Ähnliches handeln.
Darüber hinaus gibt es noch andere Gebiete wie die Archäometrie (d.\,h.\ Geophysik in Verbindung mit Archäologie) oder das Aufspüren von Körpern im Boden, wie z.\,B.\ Bomben oder Leitungen.

%\iffalse
\parshape=11
0cm 6.6cm
0cm 6.8cm
0cm 7.0cm
0cm 7.4cm
0cm 8.0cm
0cm \columnwidth
0cm \columnwidth
0cm \columnwidth
0cm \columnwidth
0cm \columnwidth
0cm \columnwidth
%\fi
Wie läuft das Geophysik-Studium nun genau ab.
Das könnt ihr ganz genau in der Prüfungsordnung nachlesen.
Diese wird auch gelegentlich mal geändert, sodass die Leute in den Semestern über euch vielleicht etwas andere Vorlesungen hören mussten als ihr.
Die Fachschaft Geophysik hat zum Glück bei den Änderungen der Prüfungsordnung ein Wörtchen mitzureden, sodass die Änderungen meist zu Gunsten der Studierenden sind.
Die aktuellste Version findet man immer als Link auf der Geophysik-Homepage:
\begin{center}
	\url{https://www.uni-muenster.de/Physik.GP}
\end{center}

Los geht es im ersten Semester mit der Vorlesung "Einführung in die Geophysik".
Darin wendet man bereits die in Physik und Mathematik erlernten Methoden auf geophysikalische Fragestellungen an.
Die Inhalte überschneiden sich dabei zwar teilweise mit denen der Vorlesung "Die Erde", welche von Professoren der geowissenschaftlichen Institute gehalten wird, was jedoch nicht weiter schlimm ist.
Man betrachtet so die gleichen Themen sowohl aus geologischer als auch aus physikalischer Sicht.
Wie man dem Studienverlauf (siehe unten) entnehmen kann, hört man jedoch genau wie alle Physik-Bachelor-Studenten das volle Physik- und Mathe-Programm in den ersten drei Semestern.
Erfahrungsgemäß nehmen diese Vorlesungen und die dazugehörigen Übungen mit Abstand am meisten Zeit in Anspruch, sodass man hier relativ viel zu tun hat.
Noch ein Tipp zu den Physik- und Mathe-Übungsgruppen: Es kann passieren, dass sich diese Termine mit Vorlesungen aus der Geophysik der den Geowissenschaften überschneiden.
Falls dies der Fall ist, sollte man versuchen, die Termine für die Übungen in Physik umzulegen (diese werden in der ersten Vorlesungswoche festgelegt).

%\iffalse
\parshape=24
%0cm \columnwidth
0cm \columnwidth
0cm \columnwidth
0cm \columnwidth
0cm \columnwidth
0cm \columnwidth
0cm \columnwidth
0cm \columnwidth
0cm \columnwidth
0.8cm \dimexpr\columnwidth - 0.8cm
1.6cm \dimexpr\columnwidth - 1.6cm
\dimexpr\columnwidth - 7.2cm 7.2cm
\dimexpr\columnwidth - 6.8cm 6.8cm
\dimexpr\columnwidth - 6.7cm 6.7cm
\dimexpr\columnwidth - 6.6cm 6.6cm
\dimexpr\columnwidth - 6.6cm 6.6cm
\dimexpr\columnwidth - 6.7cm 6.7cm
\dimexpr\columnwidth - 6.8cm 6.8cm
\dimexpr\columnwidth - 7.2cm 7.2cm
\dimexpr\columnwidth - 7.7cm 7.7cm
0cm \columnwidth
0cm \columnwidth
0cm \columnwidth
0cm \columnwidth
0cm \columnwidth
%\fi
Im zweiten Semester lernt man in "Einführung in die geophysikalische Datenverarbeitung" die in den Naturwissenschaften häufig verwendete Programmiersprache Fortran kennen.
Auch hier muss zunächst gesagt werden: Keine Panik! Auch wer noch nie unter Linux-/Unix-Systemen gearbeitet und noch nie vorher etwas programmiert hat, kommt hier klar.
Man fängt wirklich von null an.
Zusätzlich werden vom Rechenzentrum~(ZIV) in den Semesterferien etliche Kurse angeboten.
PC-Arbeitsplätze stehen jedem Studenten der Geophysik in den beiden CIP-Pools im Institut immer zur Verfügung.
Generell wird in der Geophysik sehr exzessiv am Computer gearbeitet.
Ein Grund dafür ist, dass die Datenmengen so riesig sind, dass eine andere Möglichkeit der Bearbeitung gar nicht vorhanden ist.
Der andere Grund ist, dass man in Tiefen größer als \SI{10}{\km} technisch gar nicht mehr vordringen kann und die Physik in Gegenden wie dem Erdmantel oder Erdkern durch numerische Computermodelle simuliert wird.

Außerdem beginnt man in diesem Semester mit einer Vorlesung zur angewandten Geophysik, in welcher vier verschiedene Geländemessungen durchgeführt werden und diese Vorlesung wird im drittem und viertem Semester weiter fortgesetzt.
Man lernt die theoretischen Grundlagen der typischen geophysikalischen Untersuchungsmethoden wie Gravimetrie, Magnetik, Radar oder Geoelektrik kennen.
Zwischen dem vierten und fünften Semester nimmt man am internationalen Feldkurs teil.
Dies ist eine Exkursion, in deren Genuss nur Geophysik-Studenten kommen.
Er findet in der vorlesungsfreien Zeit in Zusammenarbeit mit Universitäten aus Edinburgh und Paris statt.
Dort lernt man erneut die in den Vorlesungen behandelten Methoden mit all ihren Gerätschaften kennen.
Der Feldkurs ist darüber hinaus eine tolle Möglichkeit, Studierende aus dem Ausland kennenzulernen und Kontakte zu knüpfen.
Der Feldkurs findet immer im Wechsel in England, Frankreich oder Deutschland statt.
Ab dem vierten und fünften Semester wird es auch in der Physik etwas angewandter
Zunächst lernt man im Modul Experimentelle Übungen~I die Experimentalphysik näher kennen.
Man sitzt dabei von Woche zu Woche in einem Raum, baut ein Experiment auf, misst Daten und wertet diese in einem Protokoll aus.
Der praktische Teil der Geophysik ist dagegen deutlich erfrischender.
Außerdem werden die mathematischen Kenntnisse vertieft und auf rein geophysikalische Probleme angewandt und ab dem vierten Semester wird sich auch mehr mit der Seismologie beschäftigt, denn im vierten und fünften Semester beginnen auch die Vorlesung zu diesem Bereich der Geophysik.
Ab dem vierten Semester kann man in den Geowissenschaften wählen, welchen Schwerpunktbereich einen am meisten interessiert. 
Im fünften Semester werden  auch Kontinuumsmechanik und spezielle Anwendung von geophysikalischen Programmen gelehrt.

Im abschließenden sechsten Semester stehen die Anfertigung der Bachelorarbeit, ein Seminar sowie die, wenn nicht schon vorher erledigten, Allgemeinen Studien, an. Bei den Allgemeinen Studien kann man aus einer großen Auswahl von Kursen aus den Bereichen Planetologie, Chemie, Klimatologie, Sprachen und noch vielem mehr auswählen. Im Seminar lernt man, deutsch- und englischsprachige Vorträge über wissenschaftliche Themen zu halten. Auch hier: keine Panik! Auch wer der englischen Sprache nicht ganz so mächtig ist, wie Leute, die ein Auslandssemester verbracht haben oder Englisch-LK hatten, wird hier gut betreut. Es gibt in der Geophysik leider ohnehin nicht besonders viele deutsche Fachbücher, sodass man hier schon von Anfang an auch mal in ein englisches Buch reinschauen muss. Diese sind alle in unserer Bibliothek im Institut vorhanden und können dort eingesehen und teilweise auch ausgeliehen werden.


Ein letzter, ultimativer Tipp: Als Geophysik-Student kann man an dem \emph{Geophysikalischen Aktions-Programm}, kurz GAP, teilnehmen.
Dies ist ein jährliches Treffen, welches von und für Studenten organisiert wird und jedes Mal in einer anderen Stadt stattfindet.
Bei bis zu 100~Teilnehmern aus Deutschland, Polen und Österreich kommt beim ersten Abend auf der Icebreaker-Party internationales Flair auf.
Der zweite Tag besteht meist aus Exkursionen zu interessanten Orten in oder um die gastgebende Stadt herum.
Am dritten und letzten Tag stellen sich Institut und Professoren vor und es werden Vorträge von Wissenschaftlern und Sponsoren (also potenziellen Arbeitgebern) gehalten.
% XXX Jedes Jahr GAP-Termine/-Orte aktualisieren!
Nachdem das GAP zuletzt im September 2021 in Kiel war, findet das nächste hoffentlich 2022 in Berlin statt.

\fibelsig{Kevin}
\end{multicols}


\begin{table}[h]
    \centering
    \begin{tabular}{|c|c|c|c|c|c|c|}
        \hline
        \textbf{Semester} & \multicolumn{6}{c|}{\textbf{Module im B.\ Sc.\ Geophysik}}
        \\ \hline
       \multirow{3}{1cm}{ 1. (WiSe)} & \multirow{6}{2.3cm}{Geophysik I \footnotesize 12~LP}& \multirow{3}{2.1cm}{}&\multirow{12}{2cm}{} & \multirow{3}{1.8cm}{Physik I \footnotesize 14~LP}& \multirow{6}{2cm}{Grundlagen der Mathematik \footnotesize 16~LP} & \multirow{3}{1.9cm}{Geowissen\-schaften~I \footnotesize 4~LP} \\
       &&&&&&\\
       &&&&&&\\
        \hhline{-~~~-~-}
         \multirow{3}{1cm}{2. (SoSe)} & &\multirow{6}{2cm}{}&  &\multirow{3}{1.8cm}{Physik II \footnotesize 14~LP} & &\multirow{3}{2cm}{}\\
         & & & & & &\\
         &&&&&&\\
         \hhline{--~~---}
         \multirow{3}{1cm}{3. (WiSe)} &\multirow{6}{2.3cm}{Geophysik~II \footnotesize 13~LP} & &&\multirow{3}{1.8cm}{Physik III \footnotesize 14~LP} & \multirow{3}{2cm}{Integrations\-theorie \footnotesize 8~LP} &\multirow{3}{1.9cm}{Geowissen\-schaften~I \footnotesize 4~LP}\\
         & & & & & &\\
         &&&&&&\\
         \hhline{-~-----}
         \multirow{3}{1cm}{4. (SoSe)} & &\multirow{6}{2.3cm}{Geophysik~IV \footnotesize 9~LP}& \multirow{3}{2.3cm}{Geophysik~III \footnotesize 10~LP} &\multirow{6}{1.8cm}{Physik: Exp. Übungen~I \footnotesize 8~LP} &\multirow{9}{2cm}{Fachüber\-greifende Studien \footnotesize 10-14~LP} &\multirow{9}{1.9cm}{Geowissen\-schaften~II \footnotesize 11-15~LP}\\
         & & & & & &\\
         &&&&&&\\
         \hhline{--~-~~~}
          \multirow{3}{1cm}{5. (WiSe)} &\multirow{6}{2.1cm}{Geophysik~VI \footnotesize 7~LP} & &\multirow{3}{2.3cm}{Geophysik~V \footnotesize 9~LP}& & &\\
         & & & & &&\\
         &&&&&&\\
         \hhline{-~---~~}
         \multirow{3}{1cm}{6. (SoSe)} & &\multirow{3}{2.1cm}{Bachlor\-projekt \footnotesize 13~LP}&&&&\\
         & & & & & &\\
         &&&&&&\\
         \hline
    \end{tabular}

    \label{tab:my_label}
\end{table}

