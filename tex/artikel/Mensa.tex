% XXX Jedes Jahr Mensa-Preise prüfen!
\section[Das kleine Mensa~1~×~1]{\boldmath Das kleine Mensa~${1 \times 1}$}
\begin{multicols}{2}
\begin{quote}
	\textit{Allein zu essen ist für einen philosophierenden Gelehrten ungesund.}
	
	\hfill--- Immanuel Kant
\end{quote}

Alltäglich gegen Mittag beginnt der Magen zu knurren.
Für die Leute, die nicht jeden Tag selbst kochen wollen, am Nachmittag noch Veranstaltungen haben oder einfach soziale Kontakte pflegen wollen, gibt es die Mensa.
Das Gute ist, dass die Mensa~am~Ring direkt nebenan liegt.
Die meisten Studierenden gehen zwischen 11:45~Uhr und 12~Uhr direkt nach den Vorlesungen in die Mensa.
Dort erwartet euch vermutlich eine lange Schlange, denn um 12~Uhr ist der Anlauf dort am größten. Wer dies umgehen möchte, kann sich natürlich um 11:45~Uhr beeilen oder wenn möglich ein wenig warten. Um etwa 12:20~Uhr ist es meist schon wieder wesentlich leerer.
Doch auch die Zeit in der Schlange vergeht eigentlich recht schnell. Schwieriger kann da die Platzsuche sein, insbesondere für größere Gruppen. Gegen 12~Uhr ist die Mensa am vollsten, sodass mehr als 6 Personen Probleme bekommen können, zusammen zu sitzen, danach leert sie sich aber langsam wieder. Um etwa 12:20~Uhr wird man kaum noch Sitzplatzprobleme haben.
Bevor man sich anstellt, ist es aber hilfreich, ein bisschen über die Mensa zu wissen!

Seit dem Sommersemester 2017 benötigt ihr zum Bezahlen lediglich euren Studierendenausweis.
Das Guthaben darauf kann an den Automaten im Foyer der Mensa und mittlerweile auch online über die Website des Studierendenwerks aufgeladen werden.
Wenn das Guthaben nicht mehr ausreicht, um euer Essen zu bezahlen, könnt ihr den Ausweis auch an der Kasse noch aufladen. 
Dies ist jedoch mit einem kleinen Aufpreis behaftet und euch ist die Missgunst an der Kasse und aller in der Schlange hinter euch sicher.
Alternativ können auch Komilitonen für euch bezahlen, ihr solltet euch aber trotzdem als Studi ausweisen können, da sonst höhere Preise anfallen.
Falls ihr den Studierendenausweis noch nicht zugeschickt bekommen habt, könnte das daran liegen, dass ihr noch kein Foto dafür hochgeladen habt.
Das solltet ihr dann so bald wie möglich nachholen -- gerade, weil ihr diesen auch als Bibliotheksausweis und zum Drucken und Kopieren benötigt.

 \begin{center}
 	\includegraphics[width=.55\columnwidth]{private/res/studierendenausweis.pdf}
 \end{center}

Viel wichtiger aber stellt sich jeden Mittag erneut die Frage, was es überhaupt zu Essen gibt und was du selbst essen willst.
Dazu gibt es den Mensaplan, den ihr auf der Website der Mensa

\begin{center}
	(Link:
	\url{https://www.stw-muenster.de/de/essen-trinken/mensen/am_ring})
\end{center}

einsehen könnt. 
Dort könnt ihr den Plan entweder direkt im Browser öffnen (\url{https://muenster.my-mensa.de}) oder ihr ladet euch die zugehörige App herunter.
Normalerweise geht man aber einfach los, idealerweise mit einer Person in der Gruppe, die den Tagesplan auswendig kennt, oder man schaut auf die großen LCD-Bildschirme im Foyer.
Dort könnt ihr alle Tagesangebote sowohl "Oben" als auch im Buffetsaal "Unten" und deren Preise erfahren.
Zusatzstoffe und Allergene werden unter den Gerichten mit Zahlen bzw. Buchstaben gekennzeichnet und andere eventuell wichtige Eigenschaften des Gerichts (z.\,B. "mit Fisch", "vegetarisch", "vegan") werden mit Piktogrammen angezeigt.
Die dazugehörige Legende hängt an verschiedenen Stellen bei Essensausgabe und Kassen oder kann unter den Tagesangeboten im Online-Mensaplan eingesehen werden.

Es lohnt sich, sich vor dem Anstehen über das Angebot zu informieren, da es seit dem Sommersemester 2022 einige Änderungen im Preissystem gegeben hat (mit Vor- und Nachteilen, je nachdem, wen man fragt).
Früher wurden "Oben" immer 4~Menüs mit jeweils drei beliebigen Beilagen zu festem Preis angeboten.
Die Hauptgerichte der Menüs gibt es immer noch, wobei normalerweise mindestens eine vegetarische und eine vegane Option dabei ist, die Beilagen müssen aber individuell zusammengestellt werden.
Hier ist also Kopfrechnen gefragt, denn die Preise können ziemlich variieren.
Jedes Hauptgericht und jede Beilage ist mit eindeutigem Preis auf dem Mensaplan gekennzeichnet.
Offiziell sollte das neue System individuellere Gerichte und transparente Preise ermöglichen, durch die Hintertür erfolgte dabei aber auch eine lange überfällige Preisanpassung, die nicht gut kommuniziert wurde und dem Studierendenwerk einige Kritik eingehandelt hat.
Die Hauptgerichte liegen preislich nun bei \SI{1,10}{\euro} bis \SI{3,50}{\euro}.
Hinzu kommen so viele Beilagen wie ihr möchtet bzw. euch leisten könnt.
Zur Auswahl stehen meist Nudeln, Reis oder Kartoffeln, verschiedene Gemüseschälchen, Salate, Desserts, Obst und manchmal das Hauptgericht vom Vortag.
Die Preisspanne reicht hier von \SI{0,30}{\euro} für Nudeln und Reis bis zu \SI{1,10}{\euro} für Salate und Desserts.
Je nach Essensphilosophie können die Beilagen auf das Hauptgericht abgestimmt werden, aber auch Bratkartoffeln mit Pommes und Kartoffelbrei sind denkbar, sowie meherer Hauptgericht ohne Beilagen.

Nach einigen Mensabesuchen werdet ihr bemerken, dass die Hauptgerichte sich alle paar Wochen wiederholen.
Das ist aber gar nicht so schlimm, denn die Gerichte schmecken dann wieder genauso gut oder schlecht wie zuvor.
Bei den ersten Besuchen solltet ihr aufpassen, dass die Bilder auf den Bildschirmen (soweit vorhanden) oftmals fast nichts mit dem Aussehen des Essens zu tun haben.

%\begin{center}
%	\fibelimgtext[bottom left]{
%		\includegraphics[width=0.95\columnwidth]{res/xkcd/149_sandwich.png}
%	}{\url{https://xkcd.com/149}}
%\end{center}

"Unten" im Buffetsaal gibt es ein vielfältiges Angebot von immer wiederkehrenden Menüs.
Während der vorlesungsfreien Zeit bleibt dieser geschlossen und es gibt mehr Menüs "Oben".
Jeden Tag könnt ihr an der Grillstation Currywurst mit Pommes erwerben.
Regelmäßig gibt es den beliebten Mensaburger (auch vegetarisch) in verschiedenen Variationen, welcher so groß ist, dass sich an ihm die Geister scheiden, ob er mit Messer und Gabel oder mit der Hand gegessen werden muss.
An der Wokstation gibt es häufig Aktionen – meist aber Gerichte, die in großen Pfannen zubereitet werden können.
Darunter vermutlich am wichtigsten ist das allseits beliebte (wirklich, die Schlange ist ein Alptraum) Gyros mit Tzatziki, viel Krautsalat und Pommes, das auch vegetarisch beliebt ist, dann aber ohne den Gyrosanteil und mit mehr Kraut.
Dazu gibt es eine Pastatheke, ein Buffet mit Nudeln, Aufläufen und aufgewärmten Resten (häufig eine Anlaufstelle, wenn man sonst nichts mag) und ein Salatbuffet, an dem man sich auch Wraps zusammenstellen kann und das sich insbesondere im Sommer großer Beliebtheit erfreut.
Allerdings eine Warnung: Die Teller vom Buffet werden an der Kasse nach Gewicht abgerechnet und das wird sehr schnell mehr als beim Zusammenstellen gedacht.
Außerdem ist ein überquillender Wrap sein eigenes Problem.
Falls ihr Pommes nehmt, ist es empfohlen, nachzusalzen und gegen einen kleinen Aufpreis Majo oder Ketchup zu erstehen, Senf ist wie in einer richtigen Frittenbude kostenlos.
Des Weiteren gibt es ab und zu einen sehr günstigen Eintopf an der passend genannten Eintopfstation.
Zu jedem Essen gibt es ein recht gutes Angebot an preiswerten Getränken.

Häufig gibt es (meist an der Wokstation) verschiedene Aktionen, so etwa in der jeweiligen Saison z.\,B. Spargelgerichte oder Erdbeeren mit Schlagsahne. 
Die Mensen achten dabei auch insgesamt auf Saisonalität und Regionalität in den Gerichten, so stammen zum Beispiel die Kartoffeln ausschließlich aus dem Münsterland.
%Seit Neuerem bietet die Mensa außerdem regelmäßig Friedensteller an. Dies sind Gerichte, die nach Rezepten der Friedensteller-Intiative (überraschend, ich weiß) gekocht wurden und sich durch Nachhaltigkeit auszeichnen, z.\,B. durch Saisonalität und Regionalität der Zutaten. Diese ersetzen normalerweise eine vegetarische (oder in einigen Fällen vegane) Auswahl "Oben".

Wenn ihr fertig mit dem Essen seid, seid ihr noch lange nicht mit der Mensa fertig!
Entweder ihr holt euch noch einen Nachschlag oder ihr macht euch mit eurem Tablett auf zum Geschirrband.
An der Wand davor hängt dann eine genaue Anleitung, wie ihr euer Geschirr zu sortieren habt, damit es keinen Ärger vom Mensapersonal gibt und ihr euch nicht als unerfahrener Ersti outen müsst.

\fibelimgtext{
	\includegraphics[width=\columnwidth]{res/xkcd/720_recipes.png}
}{\url{https://xkcd.com/720}}

Falls jedoch jemand gar nichts gefunden haben sollte, könnt ihr es auch mit einer Pizza oder einer Waffel im Viva Campus-Café im Erdgeschoss versuchen.
Hier werden auch die meisten anstehenden Sportereignisse übertragen und morgens in der Wiederholung gezeigt.
Außerdem könnt ihr hier recht entspannt sitzen und etwa eine Kaffeespezialität trinken.

Dazu kommen noch diverse Läden wie ein sehr nützlicher Copyshop und Schreibwarenladen (Kopieren und Drucken ist allerdings preiswerter mit dem Guthaben des Studierendenausweis an den Kopierern z.\,B.\ im Lernzentrum oder der IG1. Aber vergesst eure Karte dort nicht!).
Außerdem gibt es noch eine Filiale der Techniker Krankenkasse, den Aster Reise Service, einen Info-Point des Studierendenwerks und zwei Geldautomaten.

Alternativ könnt ihr auch die anderen Mensen in Münster besuchen, insbesondere die Mensa~am~Aasee sei hier aufgrund der großen Auswahl an Salaten und dem grandiosen Buffet empfohlen.
Nebenan gibt es zudem das Café Hier und Jetzt, welches rein vegetarische und vegane Gerichte anbietet und sogar Abends und Samstags geöffnet ist.
Die anderen Mensen und Cafés findet ihr auch alle in der Mensa-App sowie unter \url{https://www.stw-muenster.de/essen-trinken/mensen/}.

\fibelsig{Moritz}
\end{multicols}
