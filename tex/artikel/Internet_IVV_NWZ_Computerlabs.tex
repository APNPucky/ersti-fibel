\section{Internet, IVV~NWZ, Computer\-Labs~\&~Co.}
\begin{multicols}{2}
	\fibelimgtext[below left]{
		\includegraphics[width=\columnwidth]{private/res/comics/hacker.pdf}
	}{Jan Tomaschoff\qquad© Cartoon-Caricature-Contors, Pfaffenhofen}
\subsection{WLAN für alle!}
Wo mittlerweile viele Restaurants und Cafés freien Internetzugang über WLAN zur Verfügung stellen und Parteien damit werben, das an Münsteraner Schulen fortzuführen, in einer Zeit, in der wissenschaftliches Arbeiten ohne den schnellen Austausch relativ großer Daten nicht mehr denkbar ist, gibt es natürlich auch an der Universität eures Vertrauens frei verfügbaren WLAN-Zugang für alle. Wenn sie an der WWU arbeiten oder studieren oder einer am eduroam-Projekt teilnehmenden Hochschule sind.
Außerdem bekommt ihr noch ein ganzes Paket an Cloud-Speichern und könnt im Uni-Netz sogar von überall aus drucken oder eigentlich teure Bücher und Programme herunterladen.

\subsubsection{Und wie funktioniert das?}
Ihr erhaltet bereits mit eurer Einschreibung in die Uni eine Nutzerkennung, die aus eurem Vor- und Nachnamen gebildet wird, und ein dazugehöriges Passwort.
Zusätzlich erhaltet ihr eine Uni-E-Mail-Adresse -- hängt für diese einfach \texttt{@uni-muenster.de} oder \texttt{@wwu.de} (funktioniert beides) an eure Nutzerkennung an.

Dadurch, dass ihr an der WWU Physik studiert, habt ihr einige praktische, kostenlose Angebote, die ihr mit eurer Nutzerkennung direkt nutzen könnt.

Technisch seid ihr dafür mit eurer Kennung Teil sogenannter Nutzergruppen, die euch verschiedene Berechtigungen geben. Für Studierende der Physik sind das zunächst ganz automatisch \texttt{u0dawin} (Studierende) und \texttt{p0stud} (Angehörige der Physik).

\subsection{Was kann man denn nun alles machen?}
So einiges. Der wichtigste Dienst ist dabei wohl euer E-Mail-Postfach.
Viele relevanten Infos werden euch darüber geschickt -- So zum Beispiel die Aufforderung, den Semesterbeitrag zu zahlen, oder die Information, dass eine Vorlesung kurzfristig ausfällt.
Ihr seid sogar verpflichtet, die Mails der Universität zu lesen -- Wenn ihr also den Semesterbeitrag nicht zahlt, werdet ihr sogar exmatrikuliert -- also passt auf! Auch die Erinnerung zur Prüfungsanmeldung und unseren Newsletter sowie Infomails (mit denen wir uns aber sehr zurückhalten, wir wollen euch ja nicht "zuspammen", und auch wir kennen das Problem überfluteter Postfächer :) ) erhaltet ihr per Mail an eure Uni-Adresse.

Uns, die Fachschaft eures Vertrauens, erreicht ihr übrigens über die Adresse \email{fsphys@uni-muenster.de}.

\subsubsection{Portale}
Im Portal myWWU (\url{https://www.uni-muenster.de/mywwu}) sind die wichtigsten Dienste der Uni zusammengefasst.
Dazu gehören insbesondere das E-Mail-Postfach und das Vorlesungsverzeichnis (HIS~LSF) es gibt aber auch einen Kalender (der euch auch erinnert, wann ihr Bücher zurückgeben müsst) und von euch ausgewähle Newsfeeds\dots Wenn man myWWU richtig nutzt, kann es sehr praktisch sein.

Auch sehr nützlich ist disco (\url{https://disco.uni-muenster.de}), das Suchsystem der ULB.
Hier sind der OPAC (ein Katalog- und Ausleihsystem der ULB) und viele weitere Verzeichnisse integriert, sodass ihr in vielen Millionen Dokumenten suchen könnt.
Ihr braucht also nicht jedes Mal zur ULB zu laufen, um Bücher zu verlängern.
Nochmal wichtiger sind Buch- und Literaturrecherchen, die ihr schnell und effektiv per Netz an den verschiedensten Stellen machen könnt.
Daneben gibt es über die ULB und den Fachbereich Physik auch einen kostenlosen Zugang zu allen wesentlichen wissenschaftlichen Zeitschriften und zu vielen Bücher z.\,B.\ von Springer.

Ruft dazu aus dem Uni-Netz (welches ihr auch von zu Hause aus über das sogenannte VPN der Uni erreicht) die Webseite \url{https://link.springer.com} auf und ladet das Buch eurer Wahl einfach herunter.

\begin{center}
	\includegraphics[width=\columnwidth, height=0.3\textheight]{private/res/comics/computersuechtig.pdf}
\end{center}

\subsubsection[Sichere Datencloud -- sciebo!]{Sichere Datencloud -- sciebo! \cref{internet:sciebo}}
Seit einigen Jahren bietet die WWU gemeinsam mit anderen nordrhein-westfälischen Hochschulen eine Dropbox-ähnliche Cloud an, die im Gegensatz zu anderen bekannten Online-Speicherdiensten betont nicht-kommerziell ist und viel Wert auf Datenschutz legt.
Ganz abgesehen davon, dass die Idee zur Cloud von einem Fachschaftler kam -- Danke, Markus :) -- ist Sciebo auch wegen seiner guten Umsetzung und einfacher Bedienoberfläche unbedingt empfehlenswert.
Einfach auf \url{https://www.sciebo.de} registrieren und dann mit \texttt{<nutzerkennung>@uni-muenster.de} als Benutzername und dem soeben gewählten Passwort anmelden.
Insgesamt bekommt man dort \SI{30}{\giga\byte} freien Speicherplatz, auf den man von überall aus über's Internet Zugriff hat.

\subsubsection{Spielregeln}
Ihr habt einen ungefilterten, schnellen Zugang zum Internet und damit Zugriff auf alle möglichen Angebote.
Es sollte daher nicht unerwähnt bleiben, dass trotz der großen Zahl an Mitgliedern der Uni zurückverfolgt werden kann, wer Urheberrechtsverstöße und andere illegale Aktivitäten durchführt.
Ebenfalls führt z.\,B.\ der Versand von Spam und Viren zu einer Sperrung des Netzzugangs -- Die Sanktionen bei Zuwiderhandlung sind in der Benutzungsordnung geregelt.
Dies soll euch nicht abschrecken, dennoch solltet ihr die Spielregeln kennen.

\subsubsection{ComputerLabs \& Speicherplatz}
An quasi jedem Rechner der WWU habt ihr Zugriff auf euer persönliches Benutzerkonto -- Einfach Nutzerkennung und Passwort eingeben und los geht's.
Der Fachbereich Physik hat, auf mehrere Gebäude verteilt, "ComputerLabs" eingerichtet, an denen eine große Zahl an Rechnern für euch zur Verfügung stehen.
Mit eurer Nutzerkennung könnt ihr übrigens auch die Rechner in den Labs der Biologie und Chemie benutzen -- und umgekehrt.
Dass ihr überall dieselbe Arbeitsumgebung, euer Netzlaufwerk (Laufwerk \texttt{I:} mit \SI{10}{\giga\byte} Speicherplatz) und die gleichen Programme vorfindet, dafür ist gesorgt.
Außerdem gibt es noch allgemein zugängliche ComputerLabs wie die im WWU-IT-Gebäude~'Einsteinstraße~60'.
Diese können von allen Angehörigen der Uni verwendet werden, allerdings stehen euch hier nicht dieselbe Software-Auswahl und Arbeitsumgebung zur Verfügung wie bei den Rechnern des naturwissenschaftlichen Zentrums.
Standardmäßig ist beispielsweise nur euer WWU-IT-Netzlaufwerk (Laufwerk \texttt{U:} mit \SI{4}{\giga\byte} Speicherplatz) und nicht das \texttt{I}-Laufwerk eingebunden.

Die Netzlaufwerke lassen sich auch von zu Hause aus via VPN erreichen, Anleitungen dazu gibt es auf den Webseiten von IVV NWZ und WWU IT.

Die ComputerLabs der Physik findet ihr an folgenden Orten:
\begin{description}
	\item[Angewandte Physik:] 10~Windows-PCs.
	\item[Lernzentrum:] 6 PCs
	\item[Institut~für~Kernphysik,] 2.~Stock: 11~Windows-PCs, Scanner, s/w-Laserdrucker.
	\item[Institut~für~Theoretische~Physik,] 4.~Stock: 11 Linux-PCs.
	\item[Institutsgruppe~1~(IG1):]~
		\begin{itemize}[leftmargin=1mm]
			% \item StudiBib, Erdgeschoss, Raum~13: Zwei Windows-PCs, Scanner, Farbdrucker.
			\item Institut für Technik und ihre Didaktik (Raum 220): 9~Windows-PCs, s/w-Laserdrucker.
			\item Physikalisches~Institut, 5.~Stock, Räume~504 und 520: insgesamt 9~Windows-PCs, A3-Flachbettscanner, s/w- und Farblaserdrucker.
			\item Institut~für~Festkörpertheorie, Raum~745 und 747: insgesamt 21~Windows-PCs, Scanner, s/w-Laserdrucker.
		\end{itemize}
	\item[Institut~für~Geophysik,] Corrensstr., Raum~301 und 333, je 10~Windows-PCs.
	\item[Seminar für Didaktik des Sachunterrichts]~\\(DDSU) im Leonardo-Campus~11, Raum~104: 9~Windows-PCs, s/w-Laserdrucker.
\end{description}

Die jeweiligen Kontaktadressen für Fragen und Probleme sind im jeweiligen ComputerLab bekanntgegeben.
Auf all diesen Computern ist ein sehr umfangreiches Software-Angebot installiert, sodass ihr dort direkt arbeiten könnt.
Eine Übersicht gibt es auf der Internetseite der IVV~NWZ~\cref{internet:ivvnwz}.

Der Begriff "IVV" stammt übrigens daher, dass es neben der WWU IT -- für die zentrale Bereitstellung von IT an der Uni zuständig -- an der Uni zehn sogenannte dezentrale "Informations-Verarbeitungs-Versorgungseinheiten" (IVV) gibt, die für bestimmte Bereiche zuständig sind.
Für die Fachbereiche Biologie, Chemie und Physik ist das die IVV~NWZ (IVV~4).

\subsubsection[Weitere Angebote der IVV~NWZ und der WWU IT]{Weitere Angebote der\\IVV~NWZ und der WWU IT}
Sowohl die IVV~NWZ (mit "nwzcitrix" via Uni-VPN) als auch die WWU IT (mit "rd.wwu.de") betreiben Remote-Desktop-Server, auf die ihr von Zuhause aus zugreifen könnt.
Damit habt ihr auch von Zuhause aus Zugriff auf die meiste Software und könnt damit arbeiten.
Das Stichwort "Cloud" fällt meist irgendwann in diesem Zusammenhang und sollte heutzutage jedem bekannt sein.
Etwas Ähnliches bieten IVV und WWU IT schon seit vielen Jahren allen Studierenden an.

Auch kostengünstige Druckmöglichkeiten (A4 bis A0, auch in Farbe) werden angeboten; für die Nutzung ist eine kostenlose Anmeldung bei "Print~\&~Pay" (im WWU-IT-Portal; Link im Abschnitt~"Der Ersti-Überlebenszettel", S.~\pageref{dpü} in dieser Fibel) erforderlich.
Weitere Informationen hierzu und vielen weiteren Angeboten sowie Anleitungen gibt es bei der IVV~NWZ~\cref{internet:ivvnwz} und bei der WWU IT~\cref{internet:wwuit}, sowie in der Ersti-Woche und bei uns~\cref{internet:fsphys_software}.

\subsubsection{Software für Zuhause}
Für Studierende gibt es bei der WWU IT zwei interessante Softwarepakete, die insbesondere auch für den privaten Einsatz auf den eigenen Computern vorgesehen sind.
Interessant ist vor allem die Anti-Virus-Software (Sophos), welche von der Uni für alle bezahlt wird.
Zusätzlich können viele weitere Programme, die die WWU IT betreut, auch auf dem eigenen Rechner installiert und unter einigen Bedingungen genutzt werden~\cref{internet:wwuit_software}.

Die IVV~NWZ ermöglicht zudem allen zugehörigen Studierenden den Zugriff auf das "Azure Dev Tools for Teaching"-Programm von Microsoft.
Damit habt ihr Zugriff auf fast die gesamte Software-Palette von Microsoft, ausgeschlossen sind nur einige Office-Programme. Microsoft Office-Produkte bekommt ihr allerdings auch kostenlos, und zwar über "Lizenzen" im IT-Portal, siehe \cref{internet:wwuit}.
Diese Software dürft ihr ausdrücklich auch auf privaten Rechner installieren.
Weitere Programme wie die Mathematik-Software "Mathematica" (vielen vielleicht bereits durch die -- übrigens fürs Studium häufig nützliche -- Webseite WolframAlpha \cref{internet:wolfram_alpha} bekannt) können ebenfalls unter verschiedenen Bedingungen bezogen werden -- Mathematica kann beispielsweise nur im Uni-Netzwerk (d.\,h.\ zum Beispiel von Zuhause über eine VPN-Verbindung) genutzt werden.
Weitere Infos findet ihr wieder unter \cref{internet:ivvnwz}, \cref{internet:wwuit} und \cref{internet:fsphys_software}.

\subsection{Links}
\begin{flushleft}
	\begin{fibelurl}
		\url{https://www.sciebo.de}
		\label{internet:sciebo}
	\end{fibelurl}
	\begin{fibelurl}
		\url{https://www.uni-muenster.de/NWZ}
		\label{internet:ivvnwz}
	\end{fibelurl}
	\begin{fibelurl}
		\url{https://www.uni-muenster.de/IT}
		\label{internet:wwuit}
	\end{fibelurl}
	\begin{fibelurl}
		\url{https://www.uni-muenster.de/Physik.FSPHYS/service/software}
		\label{internet:fsphys_software}
	\end{fibelurl}
	\begin{fibelurl}
		\url{https://www.uni-muenster.de/IT/Software/Uebersicht.html}
		\label{internet:wwuit_software}
	\end{fibelurl}
	\begin{fibelurl}
		\url{https://www.wolframalpha.com}
		\label{internet:wolfram_alpha}
	\end{fibelurl}
\end{flushleft}

\fibelsig{Simon, Benedikt}

\end{multicols}

\begin{center}
\fibelimgtext{
	\includegraphics[width=.6\textwidth]{res/xkcd/722_computer_problems.png}
}{\url{https://xkcd.com/722}}
\end{center}

% \begin{center}
% 	\fibelimgtext[bottom left]{
% 		\includegraphics[width=.8\textwidth]{res/xkcd/327_exploits_of_a_mom.png}
% 	}{\url{https://xkcd.com/327}}
% \end{center}
