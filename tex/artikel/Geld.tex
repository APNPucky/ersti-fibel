\section{Das liebe Geld (BAföG und Co.)}

\begin{center}
	\vspace{-0.6cm}
	\includegraphics[width=0.7\textwidth]{private/res/comics/calvin_bafoeg.png}
\end{center}

\begin{multicols*}{2}
\textbf{Geldfragen beschäftigen euch als (zum größten Teil) frischgebackene Studierende natürlich ganz besonders.
Aus diesem Grund haben wir auf den folgenden Seiten einige Informationen zum BAföG und dem Rundfunkbeitrag zusammengestellt.}

\subsection{BAföG}
Vielleicht habt ihr euch schon gefragt, ob ihr Anspruch auf Zahlungen nach dem BAföG (BundesAusbildungsförderungsGesetz) habt.
Leider ist diese Frage nicht einfach zu beantworten.
Einen ersten Überblick kann man sich gut auf der Website des Bundesministeriums für Bildung und Forschung~(BMBF) verschaffen \cref{geld:bmbf}.
Dort gibt es einen BAföG-Rechner, der einem nach Eingabe seiner Daten schon grob verrät (natürlich unverbindlich), ob man sich Hoffnungen machen darf oder nicht.
% XXX E-Mail-Adresse aktualisieren!
Auch die Beratung im Servicebüro des Amts für Ausbildungsförderung (über der Mensa am Aasee: Bismarckallee~11, E-Mail: \email{bafoeg@stw-muenster.de}) und die gebührenfreie Telefonberatung des BMBF (Tel.:~\phonenumber{0800 2236341}, Mo–Fr 8–20~Uhr) sind für den Anfang sehr nützlich.
Genaueres erfahrt ihr im Internet unter \cref{geld:studierendenwerk-ms}.

Ebenfalls kann euch die Zentrale Studienberatung~(ZSB), in einem Nebengebäude des Schlosses ansässig, in vielen Fällen helfen.
Der AStA hat in seinem Gebäude vorm Schloss eine BAföG-/Sozial\-beratungsstelle \cref{geld:asta_sozialberatung}, welche beispielsweise bei schwierigen Fällen kompetent zur Seite steht.
Dem Web, besonders der offiziellen Seite des BMBF, aber auch beispielsweise \cref{geld:studentenwerke} und \cref{geld:bafoeg-rechner} (wo es z.\,B.\ ein umfangreiches BAföG-FAQ gibt), könnt ihr ebenfalls viele Informationen und Tipps entnehmen.
% XXX Den folgenden Hinweis bitte so lange verwenden, bis die Einkommen aus Coronajahren nicht mehr für die Berechnungen relevant sind.
Außerdem findet ihr auf den genannten Seiten Informationen zu kurzfristigen Coronaregelungen (z.\,B.\ bezüglich Regelstudienzeit und Einkommenseinbußen).

Die Wahrscheinlichkeit, dass euch BAföG zusteht, ist groß.
Daher würden wir jedem dazu raten, einen Antrag auf Ausbildungsförderung zu stellen.
Die Formulare gibt es auf der Website des BMBF \cref{geld:bmbf} oder vor dem Servicebüro.
Das Ausfüllen des Antrages kostet Zeit und Nerven, insbesondere beim Erstantrag.
Es lohnt sich aber, ihn sorgfältig auszufüllen und sich Zeit dafür zu nehmen (und vor der Abgabe jede Seite für die eigenen Unterlagen zu kopieren).
Wir können auch nur davor warnen, irgendwie zu schummeln.
Die Wahrscheinlichkeit, dass sowas auffliegt, ist hoch und die Konsequenzen reichen von Rückzahlung der zu viel gewährten Zuschüsse und Darlehen bis hin zu dicken Bußgeldstrafen.
Wenn ihr euch unsicher bei irgendeiner Angabe seid (und das kann bei den verwirrenden bzw.\ teilweise schwer verständlichen Anträgen schon mal passieren), fragt lieber noch einmal nach, als dass ihr etwas Falsches angebt.
Denn was ihr einmal abgegeben habt, zählt und lässt sich nicht mehr ungeschehen machen.
Dies solltet ihr auch auf jeden Fall beherzigen, wenn ihr einen Fachrichtungswechsel anstrebt.

\subsubsection{Formulare auch im Netz}
Wer sich gerne detaillierter über die Berechnungsgrundlagen informieren will, findet dazu z.\,B.\ Möglichkeiten im Internet, etwa auf der offiziellen Seite des BMBF, wo es auch alle Formulare zum Download gibt \cref{geld:bmbf}, auf der oben genannten Infoseite des Amts für Ausbildungsförderung in Münster \cref{geld:studierendenwerk-ms} oder auf der Seite der Studentenwerke \cref{geld:studentenwerke}.
Wenn man es ganz genau wissen will, ist der Gesetzestext mit Anmerkungen, Änderungen und den sog.\ Verwaltungsvorschriften \cref{geld:bafoeg-rechner} die übersichtlichste Quelle dafür.
Aufgrund der Komplexität und der schier unendlichen Anzahl an Einzelfällen kann hier nur ein typischer und "unproblematischer" Fall geschildert werden; im Einzelfall fragt bitte unbedingt selber noch einmal beim Amt nach.

\subsubsection{Förderungshöhe}
Genug der Vorrede, nun zur Sache: Die Förderung beträgt deutschlandweit bei den aktuell gültigen Sätzen (seit August~2020) maximal \SI{861}{\euro} (eigenständig wohnende Studierende) bzw.\ \SI{592}{\euro} (bei den Eltern wohnende Studierende).
Diese Förderung setzt sich zusammen aus dem Grundbetrag inkl.\ Wohnzuschlag (\SI{752}{\euro} bzw.\ \SI{483}{\euro}) und dem optionalen KV-/PV-Zuschlag (\SI{109}{\euro}) für Studierende, die selber kranken- bzw.\ pflegeversichert sind.
Für weitere Informationen dazu verweisen wir z.\,B.\ auf die angegebenen Internetseiten.

\subsubsection{Rückzahlung}
Das BAföG besteht zu \SI{50}{\percent} aus einem Zuschuss, der nicht zurückgezahlt werden muss, und zu \SI{50}{\percent} aus einem Darlehen, welches aber nicht verzinst wird.
Allerdings ist die Höhe der BAföG-Schulden auf \SI{10000}{\euro} begrenzt; darüber hinaus bekommt ihr also einen \SI{100}{\percent}-Zuschuss.
Die Tilgungsfristen sind recht lang und beginnen 5~Jahre nach dem Abschluss bzw.\ dem Förderungsende.
Unterhalb bestimmter Einkommensgrenzen wird dies aufgeschoben oder in wenigen Fällen sogar aufgehoben.

\begin{center}
	\includegraphics[width=\columnwidth, height=0.35\textheight]{private/res/comics/newton_bafoeg.pdf}
\end{center}

\subsubsection{Voraussetzungen}
Solltet ihr gerade euer erstes Studium beginnen, noch keine 30~Jahre (35~Jahre für einen Masterstudiengang) alt sein und die deutsche Staatsbürgerschaft besitzen, seid ihr prinzipiell anspruchsberechtigt.
% XXX Jedes Jahr Datum aktualisieren!
Der Anspruch besteht ab Studienbeginn, d.\,h.\ in eurem Fall zumindest ab dem 1.~Oktober~2021, aber erst ab dem Monat, in dem ihr den Antrag gestellt habt.
Eine rückwirkende Zahlung für davor liegende Zeiträume ist nicht möglich.
Es gilt dabei jeweils eine Frist bis zum Monatsletzten.
Aus diesem Grund solltet ihr den Antrag so bald wie möglich -- spätestens bis zum 31.10.\ -- gestellt haben, sonst verschenkt ihr eure Ansprüche für den Monat Oktober.
Dafür müsst ihr dann auch noch nicht alle nötigen Unterlagen (Steuerbescheide, Kontoauszüge, Mietbescheinigungen, \dots) zusammen haben -- es reicht, wenn ihr das "Formblatt~1" fristgerecht einreicht, im Notfall reicht sogar ein formloser Antrag zur Wahrung der Frist.
Ihr habt dann maximal einen Monat Zeit, um die fehlenden Unterlagen abzuliefern (bei Nichteinhaltung wird der Antrag abgelehnt).

\subsubsection{Bestimmung der Ansprüche}
% XXX Jedes Jahr Kalenderjahr aktualisieren!
Maßgeblich für die Ermittlung eurer Ansprüche ist das Einkommen eurer Eltern (im Allgemeinen wird elternabhängige Förderung gewährt, nur in Sonderfällen ist elternunabhängige Förderung möglich) und (sofern bei euch relevant) der Ehepartnerin oder des Ehepartners im vorletzten Kalenderjahr vor Beginn des Bewilligungszeitraums (momentan also 2019) sowie euer aktuelles eigenes Einkommen und Vermögen.
Vom Einkommen werden nach Abzug von z.\,B.\ Steuern noch diverse Freibeträge abgezogen.
Unter anderem wirkt es sich auf die Freibeträge der Eltern positiv aus, wenn ihr Geschwister habt, die noch zur Schule gehen.
Auch wenn eure Eltern z.\,B.\ eure Großeltern versorgen, erhöht das eure Chancen.
Habt ihr größere Ersparnisse, so wird der über einen gewissen Freibetrag (für ledige, kinderlose Auszubildende seit 2016: \SI{7500}{\euro}, ab WiSe 20/21 dann \SI{8200}{\euro}) hinausgehende Anteil ebenfalls zu eurem Einkommen hinzugerechnet (1/12 davon pro Monat).
Schließlich wird dieses fiktive "anzurechnende Einkommen" mit eurem Bedarf, der sich aus den sogenannten "Bedarfssätzen" zusammensetzt, verglichen.
Die Differenz "$(\text{Bedarf}) - (\text{anzurechnendes Einkommen})$" ergibt dann euren BAföG-Anspruch.
Die Bedarfssätze addieren sich dabei je nach Situation zu den oben genannten maximalen Grundbeiträgen.

Wenn das "anzurechnende Einkommen" (eures und das eurer Eltern) insgesamt bei \SI{0}{\euro} landet, erhaltet ihr also diese Maximalsätze.
Wenn ihr Geschwister habt, die ebenfalls Ansprüche auf BAföG haben, so wird das "anzurechnende Einkommen der Eltern" anteilig auf euch aufgeteilt.
Falls sich abzeichnet, dass eure Eltern im aktuellen Jahr wesentlich weniger verdienen werden als im zugrunde gelegten, insbesondere bei coronabedingten Einbußen, könnt ihr einen "Aktualisierungsantrag" stellen.
Es wird dann aus aktuellen Zahlen das voraussichtliche Einkommen geschätzt und die Zahlungen erfolgen unter Vorbehalt der Rückforderung.
Doch Vorsicht: Dieser Antrag ist nicht rückgängig zu machen! Wenn sich herausstellt, dass das Einkommen eurer Eltern wider Erwarten doch höher liegt, müsst ihr leider auf einen Teil eures BAföG-Anspruchs verzichten.

So weit zur Berechnung.
Klingt kompliziert? Ist es leider auch!
\begin{tikzpicture}[remember picture, overlay]
	\node[inner sep=0, yshift=1.1cm] at (current page.center)
	{\includegraphics[width=9cm]{private/res/sparschwein_fc_koeln.png}};
\end{tikzpicture}

\subsubsection{Förderungsdauer, Studienfachwechsel}
\parshape=25
0cm \columnwidth
0cm 8cm
0cm 7cm
0cm 6cm
0cm 5.7cm
0cm 5.5cm
0cm 5.5cm
0cm 5.5cm
0cm 5.3cm
0cm 5cm
0cm 5cm
0cm 5cm
0cm 5cm
0cm 5.5cm
0cm 5.7cm
0cm 5.8cm
0cm 5.8cm
0cm 8.4cm
0cm 8.9cm
0cm \columnwidth
0cm \columnwidth
0cm \columnwidth
0cm \columnwidth
0cm \columnwidth
0cm \columnwidth
Die Förderungshöchstdauer entspricht der Regelstudienzeit in dem jeweiligen Studiengang, im Bachelor-Studiengang also sechs Semestern.
Wer danach noch den Master in Physik machen möchte, kann weitere vier Semester gefördert werden.
Dabei sind ab dem 5.~Semester Leistungsnachweise erforderlich, was bedeutet, dass ihr, um weiterhin BAföG zu bekommen, eure erforderlichen Leistungspunkte gemacht haben müsst.
Dies wird euch dann vom Prüfungsamt \cref{geld:pa_physik} oder vom Studiengangskoordinator des Fachbereichs bescheinigt.
Neu ist seit 2010, dass es nicht mehr auf die speziellen Veranstaltungen (z.\,B.\ Physik~I) ankommt, sondern nur auf die Summe der gesammelten LP.
(Letztlich entscheidet aber die Person, die die Bescheinigung über euren Fortschritt ausstellt, was sie darauf einträgt.)
Ihr müsst zudem in der Regel alle zwei Semester einen neuen Antrag stellen (Bewilligungszeitraum 1~Jahr).

Eine besondere Herausforderung stellt ein Fachrichtungswechsel dar.
So etwas solltet ihr genau durchdacht haben, denn im schlimmsten Fall verliert ihr den Anspruch auf BAföG.
In jedem Fall fordert dann das Amt für Ausbildungsförderung von euch eine Begründung dafür.
Bevor ihr irgendetwas abgebt, empfehlen wir dringend eine Beratung, beispielsweise beim AStA \cref{geld:asta_sozialberatung}.
Wenn ihr wirklich wechseln wollt, würden wir dazu raten, dies möglichst früh zu tun.
Bis zum 3.~Semester ist dies noch relativ einfach möglich (auch die Förderungsdauer wird angepasst), spätestens nach dem 3.~Semester muss ein eventueller Wechsel entsprechend begründet sein (wichtiger oder unabweisbarer Grund).
Es kann auch Schwierigkeiten verursachen, seinen Studienschwerpunkt zu verlagern.
Der Tipp also:
Bevor ihr irgendetwas an eurem Studiengang ändert, immer vorher Beratung einholen und mit eurer*eurem Sachbearbeiter*in absprechen.

\subsubsection{Nebenverdienst}
\parshape=11
0cm \columnwidth
0cm \columnwidth
0cm \columnwidth
0cm \columnwidth
0cm \columnwidth
0cm \columnwidth
2.1cm	\dimexpr\columnwidth - 2.1cm
2.6cm	\dimexpr\columnwidth - 2.6cm
3cm		\dimexpr\columnwidth - 3cm
3.3cm	\dimexpr\columnwidth - 3.3cm
3.4cm	\dimexpr\columnwidth - 3.4cm
Falls ihr in den Semesterferien ein wenig jobben wollt, ist das bis zu einer gewissen Grenze auch unproblematisch.
Im Moment dürft ihr im Bewilligungszeitraum unter Berücksichtigung aller möglichen Abzugspositionen (Werbungskosten, Sozialpauschale) monatlich \SI{450}{\euro} brutto anrechnungsfrei hinzuverdienen.
Dieser Betrag ist ein Mittelwert.
Wenn ihr also in einem Monat mehr, in anderen dafür weniger verdient, ist das kein Problem, solange ihr im gesamten Bewilligungszeitraum (1~Jahr) nicht mehr als \SI{5400}{\euro} dazuverdient.

\subsubsection{Fazit}
\parshape=6
%0cm		\columnwidth
%0cm		\columnwidth
%1.4cm	\dimexpr\columnwidth - 1.4cm
3.55cm	\dimexpr\columnwidth - 3.55cm
3.3cm	\dimexpr\columnwidth - 3.3cm
3cm		\dimexpr\columnwidth - 3cm
2.7cm	\dimexpr\columnwidth - 2.7cm
2.7cm	\dimexpr\columnwidth - 2.7cm
1cm		\dimexpr\columnwidth - 1cm
Selber ausrechnen ist fast unmöglich.
Stellt einfach den Antrag und wartet ab -- ihr habt außer etwas Zeit nichts zu verlieren! Genauere, "offizielle" Informationen, besonders auch zu Sonderfällen, entnehmt ihr bitte beispielsweise der Info-Seite unter \cref{geld:bmbf} oder den offiziellen Heftchen, die beim BAföG-Amt erhältlich sind.

\parshape=7
%3.6cm	\dimexpr\columnwidth - 3.6cm
%3.5cm	\dimexpr\columnwidth - 3.5cm
%3.4cm	\dimexpr\columnwidth - 3.4cm
0.8cm	\dimexpr\columnwidth - 0.8cm
%0.7cm	\dimexpr\columnwidth - 0.7cm
0cm \columnwidth
0cm \columnwidth
0cm \columnwidth
0cm \columnwidth
0cm \columnwidth
0cm \columnwidth
Seit Wintersemester~2019/2020 können sich Studierende an erhöhten Bedarfssätzen und Freibeträgen des BAföG erfreuen.
Im Vergleich zu vorher wurden die Bedarfssätze und der Wohngeldzuschlag um einige Prozent angehoben.
Eine dauerhafte Lösung mit kontinuierlicher Anpassung der BAföG-Sätze wird von vielen gefordert -- bisher leider ohne Erfolg.

\subsection{Befreiung vom Rundfunkbeitrag}
Seit dem 1.~Januar 2013 werden keine Rundfunkgebühren mehr eingezogen.
% XXX Aktuellen Beitragssatz überprüfen. Stand August 2021 gibt es noch Diskussionen bei den Ländern.
Stattdessen muss nun jeder Haushalt unabhängig davon, ob es Rundfunkgeräte (TV, Radio, Internet-PC) gibt oder nicht, einen Rundfunkbeitrag in Höhe von aktuell \SI{18,36}{\euro} (seit 20.07.2021) bezahlen.
Der Unterschied zwischen Gebühren und Beitrag macht diese Änderung möglich, denn bei einem Beitrag reicht die fiktive Möglichkeit der Nutzung aus, um das Geld einzutreiben; bei einer Gebühr kommt es auf die tatsächliche Nutzung an.
Der Begriff Haushalt oder Wohnung heißt im Gesetz:
\begin{quote}
	"baulich abgeschlossene Raumeinheit, die durch einen eigenen Eingang unmittelbar von einem Treppenhaus, einem Vorraum oder von außen, nicht ausschließlich über eine andere Wohnung, betreten werden kann".
\end{quote}
Ausgenommen sind nur Gartenlauben und Wohnwagen.
Auch in Zimmern in Jugendherbergen besteht keine Beitragspflicht.

Haushalt heißt nun, dass nur noch einmal pro Wohnung Beitrag fällig wird, anstatt einmal pro Person.
Wohnt ihr in einer WG, werdet ihr zunächst alle beim "ARD ZDF Deutschlandradio Beitragsservice" (Nachfolge-Einrichtung der GEZ) erfasst, daran ändert sich also erst einmal nichts.
Wenn aber eine Person aus eurer WG den Beitrag bezahlt und auf dem Formular alle Mitbewohner*innen angibt, sind diese befreit.

Die Beitragspflicht besteht grundsätzlich ab dem ersten Tag des Monats, an dem ihr in der Wohnung gemeldet seid bzw.\ ab dem ersten Tag eines gültigen Mietvertrags und sie endet mit dem letzten Tag eines Monats, in dem ihr euch bei der Stadt abmeldet oder der Mietvertrag ausläuft (Meldung beim Beitragsservice erforderlich).
Bei WGs hat dies Konsequenzen, wenn nur die den Beitrag zahlende Person auszieht: Dann muss sich jemand andereres im gleichen Monat beim Beitragsservice anmelden und ab dann den Beitrag bezahlen, ansonsten seid ihr automatisch alle wieder zahlungspflichtig.
Unterlasst ihr eine solche Meldung, begeht ihr eine Ordnungswidrigkeit, die vom Beitragsservice mit Ordnungsgeld belegt wird.

Eine grundsätzliche Befreiung ist bei Erhalt von BAföG möglich.
Der Antrag muss innerhalb von zwei Monaten nach Erhalt des Bescheids gestellt und eine Kopie dessen oder der dem Bescheid beiliegenden "Bescheinigung über Leistungsbezug zur Vorlage bei dem Beitragsservice von ARD, ZDF und Deutschlandradio" beigefügt werden.
Bei später gestellten Anträgen besteht für die Vormonate kein Anrecht auf Befreiung.
Der Antrag muss mit Erhalt eines neuen Bescheids neu gestellt werden.
Diese Befreiung hat außer für euch nur Auswirkungen auf Ehen oder eingetragene Lebensgemeinschaften; in WGs bleiben die anderen Bewohner*innen zahlungspflichtig.

Wohnt ihr in einem Studentenwohnheim, ist die Antwort schwierig.
Die Wohnheime, in denen es WGs gibt, werden wohl genauso angesehen wie andere WGs.
Das gleiche gilt für Appartements.
Schwierig ist es bei der Miete eines einzelnen Zimmers.
Hier kommt es auf die baulichen Voraussetzungen an und es ist dann eine Einzelfallentscheidung.
Bei Zimmern zur Untermiete kommt es ebenfalls auf den Einzelfall an, ob es als Wohnung oder WG angesehen werden kann.

Weitere Informationen gibt es unter \cref{geld:rundfunkbeitrag} und z.\,B.\ auch bei \cref{geld:rundfunkbeitrag_studenten}.

\vspace{-2ex}

\subsection{Allgemein}
Alle Unterlagen, die ihr abgebt, besonders beim BAföG-Antrag, solltet ihr vorher fotokopieren und die Kopien gut abheften.
Ihr habt sonst keine Möglichkeit mehr, in eure alten Anträge Einsicht zu nehmen.
Ihr werdet es spätestens, wenn ihr den Folgeantrag stellen müsst, sehr zu schätzen wissen.
Außerdem ist es wichtig, wenn es mal Probleme oder Rückfragen geben sollte.
Weiterhin seid ihr bei diesen Dingen verpflichtet, jegliche Änderungen z.\,B.\ eures Einkommens usw.\ unverzüglich mitzuteilen.
Falls ihr dies nicht tut, dürftet ihr sehr schnell "anecken" und die Sache wird wesentlich schwieriger.

\begin{flushright}
	% XXX Jedes Jahr Angaben prüfen und ggf. Zeitangabe aktualisieren!
	Alle Angaben ohne Gewähr.\\
	Stand: August~2021.
\end{flushright}

\vspace{-2ex}

\subsection{Links}
\begin{flushleft}
	\footnotesize
	\begin{fibelurl}
		\hrefurl{https://www.xn--bafg-7qa.de}{https://www.bafög.de}
		\label{geld:bmbf}
	\end{fibelurl}
	\begin{fibelurl}
		\url{https://www.stw-muenster.de/de/bafoeg-co/bafoeg}
		\label{geld:studierendenwerk-ms}
	\end{fibelurl}
	\begin{fibelurl}
		\url{https://www.studentenwerke.de}
		\label{geld:studentenwerke}
	\end{fibelurl}
	\begin{fibelurl}
		\url{https://www.bafoeg-rechner.de}
		\label{geld:bafoeg-rechner}
	\end{fibelurl}
	\begin{fibelurl}
		\url{https://www.asta.ms/sozialberatung}
		\label{geld:asta_sozialberatung}
	\end{fibelurl}
	\begin{fibelurl}
		\url{https://www.uni-muenster.de/MNFak/Pruefungsamt/physik/physikhome.html}
		\label{geld:pa_physik}
	\end{fibelurl}
	\begin{fibelurl}
		\url{https://www.rundfunkbeitrag.de}
		\label{geld:rundfunkbeitrag}
	\end{fibelurl}
	\begin{fibelurl}
		\url{https://www.studis-online.de/StudInfo/Studienfinanzierung/rundfunkbeitrag-fuer-studenten.php}
		\label{geld:rundfunkbeitrag_studenten}
	\end{fibelurl}
\end{flushleft}

\fibelsig{Markus, Simon}
\end{multicols*}
