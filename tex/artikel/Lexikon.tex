% Befehl \fibelabk fügt eine Abkürzug/einen Eintrag mit Erklärung ein
% 	Parameter #1: Abkürzung
% 	Parameter #2: Erklärung
\newcommand{\fibelabk}[2]{%
		% Verhindern, dass innherhalb eines Absätzes auf eine neue Spalte oder
		% Seite umgebrochen wird
		\interlinepenalty=10000
		% Alle Zeilen nach der ersten Zeile einrücken
		\hangindent=0.6em
		\hangafter=1
		\textbf{\boldmath#1}\hspace{1em}#2\par
}

\section{Aküverz und Lexikon}
\begin{multicols*}{2}
\textbf{Jeder kennt das Gefühl, wenn man mit scheinbar trivialen Abkürzungen überfordert ist und man sich selbst nur noch fragt, ob es nun peinlich wäre, nach dem Sinn zu fragen, oder ob man doch lieber schweigen und nicken sollte.
	Die erste Regel in solchen Situationen lautet "Don't panic!".
	Hier die wichtigsten Abkürzungen für dein Studium.}

{\footnotesize
% Abstand zwischen Paragraphen für den Rest des Artikels ändern
\setlength{\parskip}{0.6ex}

\fibelabk{1FB}{1-Fach-Bachelor}

\fibelabk{2FB/ZFB}{2-Fach-Bachelor}

\fibelabk{42}{Antwort auf die große Frage nach dem Leben, dem Universum und Allem (siehe "Per Anhalter durch die Galaxis")}

\fibelabk{AG}{Arbeitsgruppe, Arbeitsgemeinschaft}

\fibelabk{AK}{Arbeitskreis}

\fibelabk{Aküverz}{Abkürzungsverzeichnis}

\fibelabk{AP}{Angewandte Physik}

\fibelabk{AStA}{Allgemeiner~Studierendenausschuss; Interessensvertretung der Studierenden der Universität}

\fibelabk{Aus Symmetriegründen}{Abkürzung für "Diesen Beweis zu führen habe ich momentan weder Zeit noch Lust noch die Fähigkeit. Zudem würden Sie ihn ohnehin nicht verstehen. Man kann ihn aber in der einschlägigen Literatur nachschlagen."}

\fibelabk{BAföG}{Bundes-Ausbildungsförderungs-Gesetz}

\fibelabk{BaMa}{Bachelor/Master}

\fibelabk{Burschenschaft}{siehe Verbindung}

\fibelabk{C\textsubscript{2}H\textsubscript{5}OH}{Ethanol, beliebtes Genussmittel}

\fibelabk{CampusGrün}{hochschulpolitische Gruppierung}

\fibelabk{Computer}{vollkommen nutzloses Gerät, welches zur Vernichtung von Zeit entwickelt wurde; setzt immer dann aus, wenn der Artikel dringend weg muss}

\fibelabk{c.\,t.}{cum tempore; das akademische Viertelstündchen, d.\,h.\ Veranstaltungen fangen eine Viertelstunde später an}

\fibelabk{Dekan}{vertritt den Fachbereich; er wird vom Fachbereichsrat~(FBR) gewählt}

\fibelabk{DIL}{Demokratische Internationale Liste Münster, hochschulpolitische Gruppierung}

\fibelabk{DPG}{Deutsche Physikalische Gesellschaft}

\fibelabk{Dr.\ rer.\ nat.}{Doctor rerum naturalium, also ein Doktor der Naturwissenschaften}

\fibelabk{Eva}{Evaluation der Lehre; allsemesterliche Fragebogen-Umfrage, in der Studenten Rückmeldung zu ihren Lehrveranstaltungen geben können}

\fibelabk{FBR}{Fachbereichsrat}

\fibelabk{FH}{Fachhochschule}

\fibelabk{FK}{Fachschaftenkonferenz}

\fibelabk{F-Praktikum}{Fortgeschrittenen-Praktikum im Bachelor, auch: "Experimentelle Übungen für Fortgeschrittene" (offiziell: "Experimentelle Übungen~II")}

\fibelabk{FS}{Fachschaft; eigentlich alle Studierenden des Fachbereiches Physik; normalerweise versteht man unter "Fachschaft" den Fachschaftsrat~(FSR) bzw.\ die Fachschaftsvertretung~(FSV)}

\fibelabk{FSR}{Fachschaftsrat}

\fibelabk{FSV}{Fachschaftsvertretung}

\fibelabk{FT}{Festkörpertheorie; selten: Funktionentheorie}

\fibelabk{GAU}{Größter Anzunehmender Unfall; naja, und ein Super-GAU ist dann ein\dots}

\fibelabk{Grundpraktikum}{Erstes Labor-Praktikum im Bachelor, regulär ab dem dritten Semester (offiziell: "Experimentelle Übungen~I")}

\fibelabk{HFG}{Hochschulfreiheitsgesetz}

\fibelabk{HG}{Hochschulgesetz}

\fibelabk{HZG}{Hochschulzukunftsgesetz}

\fibelabk{HIS LSF}{Software ("Lehre, Studium, Forschung"), in der sich das elektronische Vorlesungsverzeichnis der Universität Münster befindet.
	Dient der Übersicht und (manchmal) der vorläufigen Anmeldung zu Lehrveranstaltungen.
	Nicht zu verwechseln mit QISPOS! (siehe dort)}

\fibelabk{HiWi}{Hilfswissenschaftler; siehe SHK und WHK}

\fibelabk{HS}{Hörsaal}

\fibelabk{HSP}{Hochschulsport; sehr günstige Angebote fast ALLER existierender Sportarten}

\fibelabk{i.\,A.}{im Allgemeinen, im Auftrag, in Arbeit u.\,v.\,m.}

\fibelabk{IG1}{Institutsgruppe~1; Hauptgebäude der Physik; IG2 gibt es nicht\dots\
	Die IG1 soll aber in den nächsten Jahren neugebaut werden.}

\fibelabk{IVV}{Informationsverarbeitungs-Versorgungseinheit}

\fibelabk{jDPG}{junge Deutsche Physikalische Gesellschaft; auf jüngere Mitglieder (z.\,B.\ Studenten) ausgerichtete Arbeitsgruppe der DPG}

\fibelabk{Jovel}{Masematte für gut, ausgezeichnet, schön, etc.\ (und 'ne Großraumdisco am Albersloher Weg)}

\fibelabk{JuSo-HSG}{Jungsozialisten-Hochschulgruppe, hochschulpolitische Gruppierung}

\fibelabk{KatHO}{Katholische Hochschule (weitere Hochschule in Münster)}

\fibelabk{KFWN}{Kommission für Forschung und wissenschaftlichen Nachwuchs}

\fibelabk{Kinderhaus}{Keine Einrichtung für Kinder, sondern ein Stadtteil im Norden Münsters}

\fibelabk{Koedukation}{gemeinsame Erziehung von Personen männlichen und weiblichen Geschlechts; soll am Physikfachbereich vielleicht auch einmal eingeführt werden}

\fibelabk{Kommilitone, Kommilitonin}{wurde als Anrede unter Studierenden gebraucht und bedeutet soviel wie Studienkollege; historisch: Waffengefährte; im neuzeitlichen wissenschaftlichen Betrieb abgelöst durch "lieber Kollege/liebe Kollegin"}

\fibelabk{KP}{Kernphysik}

\fibelabk{Kreuzviertel}{dicht bebautes Wohngebiet nördlich der Innenstadt/des Kuhviertels}

\fibelabk{KÜ}{Kanalübergang; im Norden außerhalb Münsters gelegener Freizeit- und Badetreff}

\fibelabk{Kuhviertel}{Gebiet um die Kuhstraße; lokales Maximum der Kneipenkonzentration, dementsprechend sind dort größere Mengen an Studierenden und Studierten anzutreffen}

\fibelabk{LASER}{Light Amplification (by) Stimulated Emission (of) Radiation}

\fibelabk{Leeze}{Masematte für Fahrrad}

\fibelabk{LHG}{Liberale Hochschulgruppe, hochschulpolitische Gruppierung}

\fibelabk{Linke.SDS}{SDS: Sozialistisch-demokratischer Studierendenverband; hochschulpolitische Gruppierung}

\fibelabk{Masematte}{Münsteraner "Slangsprache"}

\fibelabk{MFG}{Mitfahrgelegenheit, findet sich auf Aushängen oder bei der Mitfahrzentrale; manchmal auch: Mit freundlichen Grüßen}

\fibelabk{MP}{Materialphysik}

\fibelabk{MZ}{Münstersche~Zeitung (siehe auch WN)}

\fibelabk{na dann\dots}{wöchentliche kostenlose Zeitschrift; Inhalt im wesentlichen: Kinoprogramm, Veranstaltungshinweise, Kleinanzeigen, Mensaplan; gibt's mittwochs an tausend und einer Stelle in Münster, z.\,B.\ in der Mensa}

\fibelabk{n.\,n. oder N.\,N.}{nomen nominandum; "noch nicht bekannt, wer es machen wird"}

\fibelabk{NWZ}{Naturwissenschaftliches Zentrum}

\fibelabk{o.\,B.\,d.\,A.}{"ohne Beschränkung der Allgemeinheit", Lieblingskürzel diverser Mathe- und Physikprofs, oft auch als o.\,E.\ ("ohne Einschränkung") abgekürzt}

\fibelabk{o.\,E.}{siehe o.\,B.\,d.\,A.}

\fibelabk{OE, O-Woche, Ersti-Woche}{Orientierungseinheit für Erstsemester}

\fibelabk{PD}{Privatdozent}

\fibelabk{Per Anhalter durch die Galaxis}{für Physikstudenten unbedingt erforderliches Werk der wissenschaftlichen Literatur}

\fibelabk{PI}{Physikalisches Institut}

\fibelabk{Promenade}{Ca.\ \SI{4,5}{\km} lange Radel- und Flanierstrecke, die ringförmig um die Innenstadt verläuft und mit reichlich Grünflächen ausgestattet ist.
	Zur Erholung in der Freizeit oder als Fahrrad-Expressroute.}

\fibelabk{q.\,e.\,d.}{quod erat demonstrandum, lat.\ "was zu beweisen war"; bei fehlerhaften Beweisen auch scherzhaft als "quo errat demonstrator" (worin sich der Beweisende irrt) oder "quod est dubitandum" (was anzuzweifeln ist) gelesen}

\fibelabk{QISPOS}{wird zur Belegung von Lehrveranstaltungen verwendet.
	Alle Bachelor- und Master-Studierenden müssen ihre Lehrveranstaltungen hier belegen, um sich diese für ihr Studium anrechnen lassen zu können.
	Nein, wir wissen auch nicht, was "QISPOS" bedeuten soll.}

\fibelabk{RCDS}{Ring christlich demokratischer Studenten, hochschulpolitische Gruppierung}

\fibelabk{Repetitorium}{geraffte und zielgerichtete Wiederholung des Vorlesungsstoffs als Vorbereitung auf Klausur oder Prüfung.
	Gibt es als Unterrichtsveranstaltung, wird aber auch freiberuflich gegen Entgelt angeboten. Die Unsitte der freiberuflichen Lehre gegen Bezahlung ist in der Physik nicht verbreitet.}

\fibelabk{Rieselfelder}{Naturschutzgebiet und Vogelreservat nordöstlich von Münster, sehr schönes Ausflugsziel}

\fibelabk{Ringlinie}{Buslinien 33 \& 34, die auf dem "Ring" (ringförmige Straßenkette, Verlauf außerhalb der und um die Innenstadt) fahren und die Anbindung der äußeren Stadtteile verbessern}

\fibelabk{Rückmeldung}{ärgerliche Pflicht eines jeden Studierenden am Ende des Semesters; Zahlung der Semesterbeitrags, Übersicht der belegten Veranstaltungen}

\fibelabk{SBR}{Studienbeirat}

\fibelabk{Schloss}{wichtigstes Verwaltungsgebäude der Uni}

\fibelabk{Schont}{Masematte für Toilette}

\fibelabk{schovel}{Masematte für schlecht, unfair, ätzend, gemein, sch\dots, usw.}

\fibelabk{sciebo}{"ScienceBox", die Campuscloud (wie Dropbox, nur besser)}

\fibelabk{SHK}{Studentische Hilfskraft (z.\,B.\ Übungsgruppenleiter)}

\fibelabk{SHB}{Studentische Hilfskraft mit Bachelorabschluss. Siehe auch: SHK}

\fibelabk{SP/StuPa}{Studierendenparlament; Organ der studentischen Selbstverwaltung, setzt sich zusammen aus den von der Studierendenschaft gewählten studentischen Vertretern}

\fibelabk{Sputnikhalle}{Club im Hafenbereich (am Hawerkamp); eher Rock-/Metal-lastig}

\fibelabk{Sputte}{siehe Sputnikhalle}

\fibelabk{SR}{Seminarraum; sieht gewöhnlich aus wie ein Klassenzimmer}

\fibelabk{SS/SoSe}{Sommersemester}

\fibelabk{s.\,t.}{sine tempore, also ohne Viertelstündchen (siehe auch c.\,t.)}

\fibelabk{SWS}{Eine Semester-Wochen-Stunde ist die Zeit, die ein Fach je Woche je Semester für Vorlesungen, Übungen, Praktika etc.\ in Anspruch nimmt.
	Steht da z.\,B.\ Physik~I, Vorlesung mit Übungen, 6+4 SWS, so hat man in einer Woche 6~Stunden Vorlesung + 4~Stunden Übung in Physik~I.}

\fibelabk{TBA}{"to be announced" (wird noch bekanntgegeben)}

\fibelabk{TBBT}{The Big Bang Theory}

\fibelabk{TBD}{"to be determined" (wird noch festgelegt)}

\fibelabk{TP}{Theoretische Physik}

\fibelabk{trivial}{ganz einfache Sache, die eh keiner versteht und die deshalb auch nicht näher betrachtet wird}

\fibelabk{UKM}{Universitätsklinikum Münster}

\fibelabk{ULB}{Universitäts- und Landesbibliothek}

\fibelabk{Verbindung}{im letzten Jahrhundert geschaffene studentische Institution zur Einrichtung von Seilschaften in Industrie und Verwaltung, häufig nach archaischen Fecht- und Trinkritualen.}

\fibelabk{VV}{Vollversammlung; Veranstaltung, bei der sich alle Studierenden eines Fachbereiches versammeln, um akute Fragen zu diskutieren}

\fibelabk{WHK}{Wissenschaftliche Hilfskraft}

\fibelabk{WN}{Westfälische~Nachrichten (siehe auch MZ)}

\fibelabk{WS/WiSe}{Wintersemester}

\fibelabk{X-Viertel, $\sfrac{X}{4}$}{siehe Kreuzviertel}

\fibelabk{ZaPF}{Zusammenkunft aller Physik-Fachschaften}

\fibelabk{ZIV}{Zentrum für Informations-Verarbeitung}

\fibelabk{ZSB}{Zentrale Studienberatung, am Schloss}
}

\vfill

\fibelsig{Simon, Benedikt \& Vorgänger}
\end{multicols*}
