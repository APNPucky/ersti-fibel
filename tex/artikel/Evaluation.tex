\section{Evaluation und studentische Veranstaltungskritik}
\vspace{-3ex}
\begin{multicols}{2}
% \subsection in diesem Artikel kleiner (mit \normalsize) darstellen, sonst ist
% zu wenig Platz
\addtokomafont{subsection}{\normalsize}
% etwas weniger Platz vor/nach \subsection (wie bei \subsubsection), sonst ist
% immer zu wenig Platz
\fibelspacingsubsubsection[subsection]

\textbf{Mittlerweile werden seit 1996 am Fachbereich~Physik in den Vorlesungen Hörerbefragungen durchgeführt.
Dazu bekommen die Studierenden in allen Vorlesungen in etwa zur Semestermitte Fragebögen, mit denen sie dem Dozenten anonym eine Rückmeldung über die Stärken und Schwächen der Vorlesung geben können.}

\subsection{Es war einmal\dots}
Wir schreiben das Sommersemester~1996.
Einige Mitglieder der Fachschaft hatten sich überlegt, eine Umfrage zu entwickeln, um den Dozenten Hilfestellungen zur Verbesserung ihrer Vorlesungen zu bieten.
Dazu wurde eine Arbeitsgruppe, die VU-AG, gebildet, die nicht nur aus Studenten bestand, sondern in der auch Dozenten involviert waren.
Gemeinsam wurden Fragebögen von anderen Fachbereichen und Universitäten besorgt, diese studiert und anschließend daraus der eigene Umfragebogen zusammengeschustert.
Im Wintersemester~1996/1997 war es dann endlich so weit: zum ersten Mal wurde die VU -- auch die Vorlesungsumfrage war dem AKÜFI zum Opfer gefallen -- durchgeführt.
Anschließend begann die Auswertung, die sich als sehr aufwändig herausstellte.

Die einzelnen Bögen wurden zunächst mit Strichlisten ausgewertet und die Ergebnisse später am Computer in eine optisch ansprechende Form gebracht.
Dies dauerte jedoch so lange, dass die Ergebnisse erst kurz vor Ende des Semesters vorlagen.
Doch auch trotz solcher Anfangsschwierigkeiten wurde die Umfrage im Fachbereich -- auch von den Dozenten -- durchweg positiv aufgenommen, auch wenn die Ergebnisse nicht immer schmeichelhaft waren.
In den nächsten Jahren wurde der Fragebogen immer weiter überarbeitet und ergänzt, aber auch so umstrukturiert, dass der Großteil der Auswertung über Scanner und entsprechende Software möglich war.
So dauerte die Auswertung im Wintersemester~2005/2006 nur noch knapp zwei Wochen.

\subsection{Die VU ist tot -- es lebe die VU}
Doch es gibt noch ganz andere Veränderungen.
Vielleicht fragt sich der eine oder andere bisher, warum dort oben über dem Artikel der Name "studentische Veranstaltungskritik" steht.
Der Grund dazu: Vor einigen Jahren überlegte sich das Land NRW, dass die Evaluation der Lehrveranstaltungen doch ganz nett sei, und verpflichtete im Jahr~2000 alle Hochschulen per Hochschulgesetz, dass zukünftig alle Lehrveranstaltungen zu evaluieren seien.
Die Räder der Bürokratie setzten sich in Bewegung, so dass die Universität~Münster nach längerer Erarbeitungszeit 2005 eine Evaluationsordnung verabschiedete, welche die Einzelheiten zur "studentischen Veranstaltungskritik" regelte.
Damit sollten einige Änderungen auch auf den Fachbereich Physik zukommen.
Zum einen sollte die Hörerbefragung statt bisher nur in Vorlesungen zukünftig in allen Veranstaltungstypen stattfinden, also auch in Seminaren, Praktika etc..
Zum anderen wurde klar: Der bisherige VU-Bogen erfüllte nicht alle Anforderungen, die das Hochschulgesetz bzw.\ die Evaluationsordnung fordert, sodass dieser durch eine neuen zu ersetzen war.
Um dem VU-Bogen den letzten Todesstich zu versetzen, wurde im Auftrag eines Uni-weiten Lenkungsausschusses für Evaluation auch ein Kernfragebogen für Vorlesungen entwickelt, der zukünftig für alle Fachbereiche verpflichtend sein soll.

\vspace{-2ex}
\begin{center}
	\fibelimgtext{
		\includegraphics[width=0.8\columnwidth]{private/res/comics/wie_war_ich.pdf}
	}{\url{https://huber0vivianne11fs.blogspot.de}}
\end{center}

Somit wurde zum Sommersemester~2006 zum ersten Mal nicht mehr der eigene Bogen verwendet, sondern die Fachschaft hat die Umfrage mit dem neuen Bogen durchgeführt.
Auch einige organisatorische Dingen sind nun anders.
Zum einen ist nicht mehr die Fachschaft für die Organisation verantwortlich, sondern der Fachbereich.
Bei uns heißt das konkret, dass die organisatorischen Dinge erst nach Rücksprache mit einer Arbeitsgruppe erfolgen können, die aus einem Dozenten, einem nicht-wissenschaftlichen~Mitarbeiter, der sich um die technischen Details kümmert, und einem Studenten erfolgen kann, was mit einem Verlust an Flexibilität einhergeht.

Aber positive Seiten gibt es bei der neuen Umfrage natürlich auch: Die Universität hat zentral für alle Fachbereiche eine spezielle Software für die Auswertung der Bögen sowie neue, schnelle Einzugsscanner angeschafft.
Somit ist der Arbeitsaufwand für die Auswertung der Umfragen merkbar geringer geworden.
Aber stand im Hochschulgesetz nicht etwas von \emph{allen} Lehrveranstaltungen? Ja, dem ist so; ungeachtet der Sinnhaftigkeit müssen alle Lehrveranstaltungen evaluiert werden.
Ein erster Schritt dazu wurde bereits im WS~2005/2006 getan, als eine vom Zentralen Lenkungsausschuss für Evaluation eingesetzte Arbeitsgruppe mit Unterstützung von Vertretern aller beteiligten Fachbereiche anfing, Kernfragebögen für Seminare und Praktische Übungen zu entwickeln.

Gegen Ende des Wintersemesters folgten die ersten Pretests in ausgewählten Veranstaltungen und zum Sommersemester~2007 standen dann die nötigen Fragebögen zur Verfügung, sodass nun auch Seminare und Praktika evaluiert werden konnten.
Doch leider waren diese Fachbereich-übergreifenden Fragebögen nicht ideal für uns\dots

\subsection{Wie ist der Stand heute?}
Im Jahr~2012 wurde erstmals eine neue Version der Fragebögen verwendet.
Nach erheblicher Kritik über die ungeeigneten Fragebögen vor allem für die Seminare sollen diese die Schwächen der bisherigen ausmerzen.
Sie bestehen aus einem für alle Veranstaltungen und alle Fachbereiche gleichen Kernfragebogen, der durch Zusatzmodule an die individuellen Bedürfnisse angepasst werden kann.
So sollen unpassende Fragen nicht mehr in den Fragebögen auftauchen.
Diese Zusatzmodule wurden z.\,T.\ in Zusammenarbeit mit einigen unserer Fachschaftsmitglieder erstellt, also aus Sicht von Studierenden.

Besonders auf Grund der Kritik der Studierenden werden die Fragebögen stetig weiterentwickelt.
So wurde im Sommersemester~2015 unser bereits stark optimierte Fragebogen für laborpraktische Übungen in Zusammenarbeit mit anderen naturwissenschaftlichen Fachbereichen mit geringen Änderungen als Kernfragebogen für alle Fachbereiche ausgegeben.
Aktuell ist die Fachschaft in allen entscheidenden Institutionen für die Evaluation involviert und kann euer Feedback zur Evaluation direkt umsetzen.
Ebenso achten wir auf die Einhaltung aller Evaluationsrichtlinien, damit die Studierenden direkt von Ihrem Feedback profitieren können.

Lange Zeit wurden die Umfragen in Papierform durchgeführt. Auch wenn es vorher schon erste Versuche gab, die Evaluation digital umzusetzen, mussten im letzten Jahr durch die Coronapandemie alle Umfragen online stattfinden. So dass wir je nach Umsetzung der Lehrveranstaltungen des nächsten Semesters auf Erfahrungen aus analoger und digitaler Evaluation zurückgreifen können. Generell gilt jedoch, dass die Evaluation während der Vorlesung durchgeführt wird, um die Anzahl an Teilnehmern gegenüber einer analogen Evaluation konstant zu halten.

Wir werden uns auch in Zukunft nicht ausruhen, sondern die studentische Veranstaltungskritik mit eurer Hilfe weiterentwickeln.

\fibelsig{Lutz,Hauke}
\end{multicols}
