\section[Die junge DPG stellt sich vor]{Mehr als nur Uni -- Die junge~DPG stellt sich vor}
\begin{minipage}{0.1\textwidth}
    \includegraphics[width=\textwidth]{res/jdpg_logo.pdf}
\end{minipage}\hfill
\begin{minipage}{0.75\textwidth}
    \centering
    \textbf{Liebe Physik-Erstis der Uni Münster,\\
    auch die junge DPG möchte Euch herzlich im Kreis der Physikstudierenden begrüßen.}
\end{minipage}\hfill
\begin{minipage}{0.1\textwidth}
    \includegraphics[width=\textwidth]{res/jdpg_logo.pdf}
\end{minipage}

\begin{multicols}{2}
In Deutschland studieren rund \num{30000} Studierende Physik.
Du bist jetzt einer von ihnen.
An "Jungphysiker:innen" wie dich wendet sich die junge Deutsche Physikalische Gesellschaft~(jDPG) mit ihren Angeboten rund um das Physikstudium.

Die Ziele der jDPG sind unter anderem ein Netzwerk für Studierende in ganz Deutschland zu schaffen, sowie ein wissenschaftliches Programm zu bieten.
Des Weiteren ermöglicht die jDPG eine umfangreiche Berufsvorbereitung und fördert den Dialog zwischen den Generationen.
Dabei versteht sie sich, zusammen mit den Physikfachschaften, als eine bundesweite Interessenvertretung von Physikstudierenden.

Mehr als \num{3500} Mitglieder zählt die jDPG derzeit.
Für diese Mitglieder und auch für alle anderen Physikstudierenden gibt es unsere Angebote sowohl deutschlandweit als auch vor Ort in den einzelnen Regionalgruppen.

Die Regionalgruppe Münster veranstaltet Exkursionen zu physikalisch interessanten Zielen (Forschungsinstitute, Unternehmen, etc.).
So ging es in den letzten Jahren z.\,B.\ zum weltweit ersten Supraleiterkabel, das im öffentlichen Stromnetz integriert ist (AmpaCity), zum Forschungszentrum Jülich und für zwei Tage nach Hamburg zum Deutschen Elektronen-Synchrotron (DESY), dem "Center for Free-Electron Laser Science" und Philips. Die letzte Exkursion ging zum Deutschen Zentrum für Luft- und Raumfahrt (DLR) in Köln.

Eine weitere Veranstaltungsreihe in Münster ist \textbf{Doctor's Diaries}, welche ca. dreimal im Semester stattfindet. Hier werden Doktorand:innen aus den unterschiedlichen Arbeitsgruppen der Physik eingeladen, um in einem lockeren Vortrag sich selbst und das Doktorprojekt vorzustellen. Wir sorgen für Kekse und Saft und ihr könnt Fragen stellen und so die aktuellen Forschungsgebiete kennen lernen. Wenn es sich anbietet, gibt es auch eine Führung durch die Labore. 

Jährlich findet ein \textbf{Preisträgertreffen} für die Abiturpreisträger:innen statt -- eine gute Möglichkeit sich untereinander kennen zu lernen und neue Kontakte auch aus anderen Semestern hier in Münster zu knüpfen.


Die Regionalgruppe Münster trifft sich mehrfach im Semester zum Stammtisch, zu dem auch du gerne vorbeischauen kannst, natürlich unverbindlich. ;)


Aktuelle Informationen zu unseren Veranstaltungen erhaltet ihr über unsere Internetseite (\url{http://www.muenster.jdpg.de}) oder die Informationskanäle der Fachschaft.

Von den deutschlandweiten Veranstaltungen ist die Sommerexkursion das Highlight für die Mitglieder der jDPG.
Fünf Tage lang treffen sich Physikstudierende aus ganz Deutschland in einer Stadt, um Einblicke in die aktuelle Forschung und Entwicklung zu bekommen, und zwar nicht im Hörsaal, sondern direkt am Schauplatz des Geschehens.
Neben dem fachlichen Teil ist aber auch eine Kneipentour und ein gemeinsames Grillen ein fester Bestandteil des Programms.

Besonders wichtig für die Kommunikation und Organisation der jDPG ist die jährliche Mitgliederversammlung.
Dort werden wichtige Beschlüsse gefasst und der gesamte jDPG-Vorstand wird neu gewählt.
Außerdem informieren der überregionale jDPG Newsletter, das Physik~Journal, die Mitgliederzeitschrift der Deutschen Physikalischen Gesellschaft, zusätzlich.


Der Anstoß zur Gründung der jDPG kam im Jahr~2005 aus der Deutschen Physikalischen Gesellschaft selbst.
Diese hatte zwar 30~Prozent studentische Mitglieder, aber die Angebote für diese junge Zielgruppe fehlten.
Als die DPG auf die Studierenden der TU~Dresden zuging, haben sich spontan fünf Jungphysiker:innen gefunden, die bereit waren, diese Lücke zu schließen.
"Wir haben das damals für ein kleines Versuchsprojekt gehalten", erzählt jDPG-Mitglied René Pfitzner.
Mit der rasanten Entwicklung der vergangenen Jahre habe damals niemand gerechnet.

Heute ist die jDPG mit über 30~Regionalgruppen eine bundesweit aktive Organisation mit festem Programm.

Weitere Informationen findest du auf
\begin{center}
    \textbf{\url{http://www.jdpg.de}}.
\end{center}

Die Angebote der Regionalgruppe~Münster und den Termin für den nächsten Stammtisch findest du auch direkt unter
\begin{center}
    \textbf{\url{http://muenster.jdpg.de}}.
\end{center}

Bei weiteren Fragen kannst du dich auch gerne an uns wenden:
\textbf{\email{muenster@jdpg.de}}.

\fibelsig{Eure Regionalgruppe Münster}

\begin{center}
    \includegraphics[width=\columnwidth]{res/jdpg_foto.png}
\end{center}

\end{multicols}

\vfill

\begin{center}
    \fibelimgtext{
        \includegraphics[width=\textwidth]{res/xkcd/669_experiment.png}
    }{\url{https://xkcd.com/669}}
\end{center}

\vfill
