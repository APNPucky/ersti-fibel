\section{Was tun, wenn die Eltern kommen?}
\begin{multicols}{2}
Nein, damit ist nicht gemeint, wie am schnellsten die Rollläden runtergelassen und die Tür verriegelt werden können\dots\
Auch nicht, welcher Arzt euch am schnellsten bescheinigt, dass ihr keinen Besuch empfangen dürft.
Nein, gemeint ist hier ein Ratgeber, wie ihr den Besuch eurer Eltern so gestalten könnt, dass er für beide Seiten schön werden kann.

\begin{center}
	\fibelimgtext{
		\includegraphics[width=\columnwidth, height=0.22\textheight]{res/xkcd/441_babies.png}
	}{\url{https://xkcd.com/441}}
\end{center}

Für eure Eltern ist es anfangs schwer, euch gehen zu lassen.
Deswegen möchten sie das Gefühl haben, weiterhin an eurem Leben teilnehmen zu dürfen.
Wenn sich eure Eltern ankündigen, plant also nicht nur ein Mittagessen in einem spießigen Restaurant, in das ihr sonst nicht reingehen würdet.

Es bietet sich immer an, die Eltern mit zum Wocheneinkauf zu nehmen.
Dann haben sie das Gefühl, mitzubekommen, was ihr macht und ihr habt Glück und müsst in den meisten Fällen nicht selbst bezahlen und könnt Getränke bequem im Auto transportieren.
Auch in die Innenstadt könnt ihr eure Eltern mitnehmen und dabei erzählen, wo ihr abends feiern geht und wo die Bibliothek ist.

Es gibt auch die Möglichkeit, bei einer Stadtrundfahrt die Innenstadt kennenzulernen.
Bei schönem Wetter gibt es viel Auswahl: Ein Ausflug in den Zoo, an den Aasee, den botanischen Garten oder zum Wandern in das Vogelschutzgebiet in den Rieselfeldern.
Bei schlechtem Wetter könnt ihr mit ihnen verschiedene Museen besichtigen, z.\,B.\ das Picassomuseum oder das Naturkundemuseum.
Im Sommer findet einmal im Monat der Promenadenflohmarkt statt.
Auch hier könnt ihr besonders bei den meisten Müttern punkten.

Die Stadt, die Hochschulen, das Studierendenwerk, studentische Kulturgruppen, Kinos, Museen und Theater bieten einmal im Jahr ein Besuchswochenende zum Studententarif an – mit Eltern-Uni, Mensa-Brunch und einer Prise Selbstironie.
Das ist die optimale Veranstaltung, um eure Eltern mit zu einem Abendprogramm mitzunehmen, ohne dabei von euren Kommilitonen dumm angeguckt zu werden.

\begin{center}
	\includegraphics[width=\columnwidth, height=0.28\textheight]{res/smbc/2016-05-15_paternity-test.png}
\end{center}

Wenn eure Eltern eine längere Anfahrt haben, solltet ihr euch Gedanken über die Unterbringung für die Nacht machen.
Habt ihr genug Platz und Lust, die Eltern bei euch übernachten zu lassen? Oder wollt ihr lieber ein Zimmer in einem nahe gelegenen Hotel buchen?

\fibelsig{Judith}
\end{multicols}

\begin{center}
	\fibelimgtext{
		\includegraphics[width=\textwidth, height=0.16\textheight]{res/xkcd/674_natural_parenting.png}
	}{\url{https://xkcd.com/674}}
\end{center}
