\section{Nachhilfe?!}
\begin{multicols}{2}
Vielleicht denkt ihr euch jetzt: Nachhilfe -- Die brauch' ich nicht, denn ich war einer der Besten in der Schule. Das wünschen wir euch allen auch. Aber manchmal trifft man während des Studiums auf Aufgaben oder Vorlesungsinhalte, die man alleine oder in der Lerngruppe -- die ihr hoffentlich schnell findet -- nicht gelöst bekommt. In solchen Fällen ist es hilfreich, wenn man jemanden Erfahrenes fragen kann und der das einem in Ruhe nochmal erklärt.

Andere wollen sich etwas nebenbei verdienen und Nachhilfe geben.

Um die Suche nach geeigneten Nachhilfelehrern zu vereinfachen, haben wir einen Learnweb-Kurs erstellt, in dem Studierende ihre Hilfe für verschiedene Fächer anbieten. Die meisten dieser Studis sind in höheren Semestern oder sogar schon im Master. Wenn ihr selbst Nachhilfe (z.\,B.\ für Schüler) anbieten wollt, könnt ihr euch einfach mit auf der Webseite eintragen und wir leiten Anfragen an euch weiter.

Den Kurs findet ihr unter dem Namen \textit{fsphysnachhilfe} oder unter \url{https://sso.uni-muenster.de/LearnWeb/learnweb2/course/view.php?id=46153}.

\fibelsig{Alex, Benedikt B.}
\end{multicols}
